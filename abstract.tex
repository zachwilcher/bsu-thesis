\chapter*{Abstract}
{\bf THESIS:} Graph Configuration Spaces and Braid Groups
\\{\bf STUDENT:} Zach Wilcher
\\{\bf DEGREE:} Master of Science
\\{\bf COLLEGE:} Sciences and Humanities
\\{\bf DATE:} May 2026
\\{\bf PAGES:} \pageref{LastPage}
\\
Given any space \(X\), the \(n\) point configuration space \(\Conf_n(X)\) is 
\(X^n - \Delta\), where \(\Delta\) is the diagonal in \(X^n\). 
Contrast this with the \(n\) point unordered configuration space \(\UConf_n(X) = \Conf_n(X) / S_n\)
where the quotient is obtained by permuting the coordinates of points in \(\Conf_n(X)\).
The fundamental groups of \(\Conf_n(X)\) and \(\UConf_n(X)\) are respectively known as the \(n\)-strand pure braid group
\(P_n(X)\) and the \(n\)-strand braid group \(B_n(X)\).
The base point will be omitted since the considered spaces are path connected.

\(B_n(\mathbb{R}^2)\) is also known as the Artin braid group and has applications
in motion planning, fluid mechanics, and quantum information theory.
Ghirst and Koditschek showed in 1999 that using a graph \(\Gamma\) as the underlying space has 
been applied to modeling the paths of automated guided vehicles.

Topologically, \(\Conf_n(\Gamma)\) is an interesting space in its own right. 
Abrams showed in 2000 that \(\Conf_2(\Gamma)\) is a surface if and only if \(\Gamma\)
is the complete \(5\) graph \(K_5\), or the bipartite graph \(K_{3,3}\).
By constructing the neighborhoods of each vertex in the configuration space
and computing the Euler characteristic, it can be shown
whether the space is a surface and what kind of surface it is.
The goals for this project are to 
find new examples of surfaces arising from \(\Conf_n(\Gamma)\) when \(n > 2\),
and to determine precisely which graphs yield surfaces when \(n > 2\).




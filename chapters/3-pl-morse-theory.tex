\chapter{PL Morse Theory}

\section{Background}

This is a modified proof of a statement in \cite{appiah2024algebraicstructurehyperbolicgraph}.
\begin{thm}
\label{thm:descendinglinks}
Let \(X\) be a finite cubical complex,
\(f: X \rightarrow \mathbb{R}\) be a PL Morse function,
and \([a,b]\) be contained in the image of \(f\). 
For any \(c\) in the open interval \((a,b)\), if
\begin{enumerate}
    \item \(f^{-1}([a,c])\) is connected.
    \item \(f^{-1}((c,b))\) contains no vertices.
    \item \(f^{-1}(\{b\})\) is a set of vertices.
    \item The descending links of each vertex \(v\) in \(f^{-1}(\{b\})\) 
        is homotopically equivalent to \(n_v\) distinct points.
\end{enumerate}
then, \(f^{-1}[[a,b]]\) is homotopy equivalent to a wedge of \(f^{-1}[[a,c]]\)
with exactly \(\displaystyle r = \sum_{v \in f^{-1}(\{b\})} (n_v - 1)\) circles and consequently
\(\pi_1(f^{-1}[[a,b]]) = \pi_1(f^{-1}[a,c])\,*\,\mathbb{F}_r\)
where \(\mathbb{F}_r\) is a free group on \(r\) generators.
\end{thm}

\begin{proof}
We first consider the case when there is only one vertex \(v\) in \(f^{-1}(\{b\})\).
Let the descending link of \(v\) be the disjoint spaces \(\{A_1, ..., A_{n_v}\}\). 

By Proposition 2.7 in \cite{Bestvina2008PL}, \(f^{-1}([c,b])\) is homotopically equivalent
rel \(f^{-1}(\{a\})\) to the cone on the descending links of \(v\) attached to \(v\). 

Since \(f^{-1}([a,c])\) is path connected, 
we can find paths from a point in each \(A_i\) to some \(p\) in \(f^{-1}(\{c\}) \cap A_1\).
Let \(Q\) be the union of the cone and these paths.

% Visually this next line is clear but I should say more.
Notice \(Q\) homotopically equivalent to a wedge of \(n_v - 1\) circles.

Since each added path in \(Q\) contracts to \(p\), we can retract \(Q \cap f^{-1}([a,c])\) to \(p\).

% I should be more explicit about how the fundamental group of the intersection is null homotopic
% and how \pi_1 of a wedge of circles is just a free group on how many circles there are.
The Seifert-Van Kampen theorem guarantees that \(\pi_1(f^{-1}([a,b]))\) is just \(\pi_1(f^{-1})([a,c]) * \mathbb{F}_{n_v - 1}\).

When there are multiple points in \(f^{-1}(\{b\})\), as mentioned in \cite{Bestvina2008PL}
the cones on each point \(v \in f^{-1}(\{b\})\) get attached to each \(v\) in a pairwise disjoint fashion.
Similar to before, by finding paths through \(f^{-1}([a,c])\) we can connect together each cone
resulting in a wedge of \(r = \sum_{v} (n_v - 1)\) circles.
The Seifert-Van Kampen theorem then guarantees that
\(\pi_1(f^{-1}([a,b])) = \pi_1(f^{-1}([a,c])) * \mathbb{F}_r\).

\end{proof}

\section{Free group factor of Pulsar Graph}

In \cite{appiah2024algebraicstructurehyperbolicgraph} it was determined that the 3-strand braid group of the ``pulsar graph''
is the free product of the 3-strand braid group on the generalized theta graph and some finite free group \(\mathbb{F}\).
However, in their analysis they did not keep track of the number of generators.
Determining the exact number of generators of this free group would be a useful result for applying
results from geometric group theory to graph configuration spaces.

To compute this free group we analyze the descending links of vertices in \(\DConf_3(\mathcal{P}_{m,n_1,0})\).
Consider a vertex of the form \(\{c_i', v, w\}\).
If neither \(v\) nor \(w\) is \(c_i\)
then a particle at \(c_i'\) can move down to \(c_i\).
The downward movement of particles at \(v\) and \(w\) then correspond to a cone for the descending link.

Consider instead vertices of the form \(\{c_i', c_i, w\}\).
If \(w \neq a_1\) and \(w \neq c_j\) for any \(c_j\),
again the downward movement of \(w\) results in the descending link being either a cone or a point (if \(w = a_2\)).
If \(w = a_1\) then the descending link of each of these \(n_1\) vertices consists of \(m\) points,
meaning there are \(n_1 (m - 1)\) generators in \(\mathbb{F}\) corresponding to these vertices.
Otherwise, if \(w = c_j\) for some \(j \neq i\) then the descending link consists of \(2\) points which puts \(n_1 (n_1 - 1)\) additional generators in \(\mathbb{F}\).
So, vertices of the form \(\{c_i', c_i, w\}\) put \(n_1 (m-1) + n_1 (n_1 - 1)\) generators in \(\mathbb{F}\).

Finally, we consider vertices of the form \(\{c_i, v, w\}\) where \(v\) nor \(w\) is \(c_j'\) for any \(j\).
Without a loss of generality, suppose \(h(v) \ge h(w)\).
Notice that if \(h(v) \le 1\) then a particle at \(c_i\) can move down to \(a_1\) without issue meaning the descending link is just a cone.
So, suppose \(h(v) > 1\).
We now proceed by cases on where \(v\) can be.

\begin{enumerate}
    \item If \(v = a_1\) and \(w = a_2\), there are \(n_1\) vertices of this form.
    Since a particle at \(a_1\) can then move down one in \(m\) ways, the descending link is \(m\) points.
    So, there are \(n_1 \cdot (m - 1)\) generators in \(\mathbb{F}\) from these vertices.

    \item If \(v = a_1\) and \(w = b_j\) for some \(j \in \{1, \cdots, m\}\), 
    there are \(n_1 \cdot m\) vertices of this form. 
    Since a particle at \(b_j\) can move down simultaneously as one at \(a_1\), 
    the descending link for each of these vertices is just a cone which contributes no generators to \(\mathbb{F}\).

    \item If \(v = c_j\) for some \(i \neq j\) and \(w = a_2\), 
    there are \(\binom{n_1}{2}\) vertices of this form.
    Since only one particle at either \(c_i\) or \(c_j\) can move down, the descending link for each of these vertices is just two points.
    So, there are \(\binom{n_1}{2}\) generators added to \(\mathbb{F}\) in this case.

    \item If \(v = c_j\) for some \(i \neq j\) and \(w = b_j\) for some \(j \in \{1,\cdots, m\}\), 
        there are \(\binom{n_1}{2} \cdot m\) vertices of this form.
        Since a particle at \(b_j\) can move down simultaneously as one at \(a_1\), the descending link
        for each vertex is just a cone contributing no generators to \(\mathbb{F}\).

    \item If \(v = c_j\) for some \(i \neq j\) and \(w = a_1\), 
        there are \(\binom{n_1}{2}\) vertices of this form.
        Since only one particle at \(w = a_1\) can move down to one of \(m\) positions, 
        there are \(\binom{n_1}{2}(m - 1)\) generators added to \(\mathbb{F}\) in this case.

    \item If \(v = c_j\) for some \(i \neq j\) and \(w = c_k\) for some \(k \neq i\) and \(k \neq j\),
        there are \(\binom{n_1}{3}\) vertices of this form.
        Since only one particle at \(c_i\), \(c_j\), or \(c_k\) can move down to \(a_1\),
        there are \(\binom{n_1}{3}(2)\) generators added to \(\mathbb{F}\) here.
\end{enumerate}
Adding together these expressions we find that vertices of the form \(\{c_i, v, w\}\) where \(v\) nor \(w\) is \(c_j'\)
contribute 
\(\frac{m n_{1}^{2}}{2} + \frac{m n_{1}}{2} + \frac{n_{1}^{3}}{3} - n_{1}^{2} - \frac{n_{1}}{3}\)
generators to \(\mathbb{F}\).

Including the other possible vertices from before we find that \(\mathbb{F}\) has
\(\frac{m n_{1}^{2}}{2} + \frac{3 m n_{1}}{2} + \frac{n_{1}^{3}}{3} - \frac{7 n_{1}}{3}\)
generators.

% TODO Explain the flipping to get the other generators

So, the number of generators total is
\[
\frac{m n_{1}^{2}}{2} + \frac{3 m n_{1}}{2} + \frac{m n_{2}^{2}}{2} + \frac{3 m n_{2}}{2} + \frac{n_{1}^{3}}{3} - \frac{7 n_{1}}{3} + \frac{n_{2}^{3}}{3} - \frac{7 n_{2}}{3}
\]
Referring to Theorem \ref{thm:pulsargrapheuler}, 
notice that the above expression is the same as \(\chi(\DUConf_3(\Theta_m)) - \chi(\DUConf_3(\mathcal{P}_{m,n_1,n_2}))\).


\begin{cor}
    \label{cor:eulerfreegroup}
    Let \(X\) be a finite cube complex,
    \([a,b]\) be the image of a PL Morse function \(f: X \rightarrow \mathbb{R}\),
    and \(A \subseteq X\).
    If
    \begin{enumerate}[label=(\roman*)]
        \item \(X\) and \(A\) are connected.
        \item \(f^{-1}([a,c]) = A\) for some \(c \in (a,b)\).
        \item The descending link of every vertex in \(X \setminus A\) is homotopically equivalent to a finite set of points. 
    \end{enumerate}
    then,
    \(\pi_1(X) = \pi_1(A)\,*\, \mathbb{F}_r\) where \(\mathbb{F}_r\) is a free group on \(r = \chi(A) - \chi(X)\) generators.
\end{cor}
\begin{proof}
    % This first sentence might want some elaboration. My logic is that if you pick some point in this non-empty
    % set and go up from it you have to hit a vertex.
    Note that \(X \setminus A\) must contain vertices since \(f^{-1}((c, b])\) is non-empty.
    Let \(c_0, c_1, ..., c_k\) be the ``heights'' of each vertex in \(X \setminus A\).
    Specifically, let \(c_0 = c\) and 
    inductively define \(c_i = \min\left(\{f(v) \mid v \in X \setminus A\}\setminus\{c_j \mid j < i\}\right)\).
    
    Note that when we write \(\chi(f^{-1}[c_{i-1}, c_i])\) or \(\chi(f^{-1}(c_i))\), 
    its understood we are making the computation in an appropriately subdivided \(X\) 
    which agrees with the original complex since subdivision does not change the Euler characteristic.

    For each \(i \ge 1\) define \(\Delta \chi_i = \chi(f^{-1}([c_{i-1}, c_i])) - \chi(f^{-1}(\{c_{i-1}\}))\).
    We claim that 
    \begin{equation}
     \label{eq:eulerfreegroup1}   
        \chi(X) = \chi(A) + \sum_{i=1}^k \Delta \chi_i
    \end{equation}
    After applying the inclusion exclusion principle we have
    \[
    \chi(X) = \chi(A) + \chi(f^{-1}([c, c_k])) - \chi(f^{-1}(\{c\})).
    \]
    Now we show \(\chi(f^{-1}([c, c_k])) - \chi(f^{-1}(\{c\})) = \sum_{i=1}^k \Delta \chi_i\) inductively.
    If \(k = 1\) the formula holds immediately. So suppose the formula holds for some \(k - 1 \ge 1\).
    After applying the inclusion-exclusion principle,
    \[
    \chi(f^{-1}([c, c_k])) = \underbrace{\chi(f^{-1}([c, c_{k-1}]))}_{\sum_{i=1}^{k-1}\Delta \chi_i} + \underbrace{\chi(f^{-1}([c_{k-1}, c_k])) - \chi(f^{-1}(\{c_{k-1}\}))}_{\Delta \chi_k}.
    \]
    Therefore, Equation \ref{eq:eulerfreegroup1} holds. Next, we claim that
    \begin{equation}
     \label{eq:eulerfreegroup2}   
        \Delta \chi_i = \sum_{v \in f^{-1}(\{c_i\})} (1 - n_v)
    \end{equation}
    where \(n_v\) is the number of retractable spaces in the descending link of \(v\).
    
    Suppose that there is only one \(v\) vertex in \(f^{-1}(\{c_i\})\). 
    Proposition 2.7 in \cite{Bestvina2008PL} guarantees that 
    \(f^{-1}[c_{i-1}, c_i]\) is homotopically equivalent rel \(f^{-1}(\{c_{i-1}\})\) to \(f^{-1}(\{c_{i-1}\})\)
    with the cone on \(\DLk(v)\) attached. 
    Let \(\epsilon > 0\) such that \(f^{-1}([c_{i-1}, c_i - \epsilon])\cup \cone(\DLk(v)) \simeq f^{-1}([c_{i-1}, c_i])\).
    Appropriately subdividing, we can then write \(\chi(f^{-1}([c_{i-1}, c_i]))\) as
    \[
        \chi(f^{-1}([c_{i-1}, c_i - \epsilon])) + \chi(\cone(\DLk(v))) - \chi(f^{-1}([c_{i-1}, c_i - \epsilon]) \cap \cone(\DLk(v))).
    \]
    Proposition 2.4 in \cite{Bestvina2008PL} gurantees \(f^{-1}([c_{i-1}, c_i - \epsilon])\) is homotopic to \(f^{-1}(\{c_{i-1}\})\).
    Since a cone is homotopically equivalent to a single point and \(f^{-1}([c_{i-1}, c_i - \epsilon]) \cap \cone(\DLk(v))\) has \(n_v\) contractible components,
    we have
    \[
    \chi(f^{-1}([c_{i-1}, c_i])) = \chi(f^{-1}(\{c_{i-1}\})) + 1 - n_v.
    \]
    So, \(\Delta\chi_i = \chi(f^{-1}([c_{i-1}, c_i])) - \chi(f^{-1}(\{c_{i-1}\})) = 1 - n_v\).
    If there is more than one vertex in \(f^{-1}(\{c_i\})\), then as remarked in \cite{Bestvina2008PL}
    the cones on the descending links are attached in a pairwise disjoint fashion.
    So, for sufficiently small \(\epsilon\) we have 
    \[
        f^{-1}([c_{i-1}, c_i]) \simeq f^{-1}([c_{i-1}, c_i - \epsilon]) \cup \underbrace{\left(\bigsqcup_{v \in f^{-1}(\{c_{i-1}\})}\cone(\DLk(v))\right)}_{C}.
    \]
    Hence,
    \(\chi(f^{-1}([c_{i-1}, c_i])) = \chi(f^{-1}([c_{i-1}, c_i - \epsilon])) + \chi(C) - \chi(f^{-1}([c_{i-1}, c_i - \epsilon]) \cap C).\)
    Since \(C\) is homotopically equivalent to the number of vertices in \(f^{-1}(\{c_{i-1}\})\)
    and \(f^{-1}([c_{i-1}, c_i - \epsilon]) \cap C\) has \(\sum_{v \in f^{-1}(\{c_{i-1}\})} n_v\) contractible components,
    Equation \ref{eq:eulerfreegroup2} holds.
    Substituting Equation \ref{eq:eulerfreegroup2} into Equation \ref{eq:eulerfreegroup1}, we have
    \[
        \chi(A) - \chi(X) = \sum_{i=1}^k \sum_{v \in f^{-1}[c_{i-1}, c_i]} (n_v - 1).
    \]
    Finally, applying Theorem \ref{thm:descendinglinks} for each closed interval \([c_{i-1}, c_i]\), 
    we obtain the exact same expression for \(\mathbb{F}_r\).
\end{proof}

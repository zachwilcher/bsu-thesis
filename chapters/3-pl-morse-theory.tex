\chapter{PL Morse Theory}
In \cite{appiah2024algebraicstructurehyperbolicgraph} it was determined that the 3-strand braid group of the ``pulsar graph''
is the free product of the 3-strand braid group on the generalized theta graph and some finite free group \(\mathbb{F}\).

\section{Computing the Euler characteristic of a graph configuration space}
To compute the euler characteristic of an unorderd \(n\) point graph configuration space we need to count
for each \(0 \le k \le n\) the number of \(k\)-cubes present.
Without further mention \(E = E(\Gamma)\) and \(V = V(\Gamma)\).

A \(k\) cube exists if and only if \(k\) particles are free to move simultaneously while \(n - k\)
particles remain fixed.
Hence the general formula for the number of \(k\) cubes in \(\DUConf_n(\Gamma)\) is
\[
    \phi(k) \cdot \begin{pmatrix}\abs{V} - 2k \\ n - k\end{pmatrix}
\]
where \(\phi(k)\) is the number of ways to choose \(k\) disjoint edges in \(\Gamma\).
\(\phi(k)\) is also known as the number of ``matchings'' of size \(k\) in \(\Gamma\).
Note that \(\phi(0) = 1\) and \(\phi(1) = \abs{E}\).

To compute \(\phi(k)\) for \(k > 1\) it is useful to consider the following set.
\begin{defn}
\(\delta(e)\) is the set of edges incident to an endpoint of an edge \(e\) in \(\Gamma\) not including \(e\) itself.
\end{defn}

\begin{figure}[h!]
    \centering
\begin{tikzpicture}
    \node (v1) at (-1, 1) {\(v_1\)};
    \node (v2) at (0, 1) {\(v_2\)};
    \node (v3) at (1, 1) {\(v_3\)};
    
    \node (v4) at (0, 0) {\(v_4\)};
    \node (v5) at (0, -1) {\(v_5\)};
    \draw (v1) -- (v2);
    \draw (v2) -- (v3);
    \draw (v2) -- (v4);
    \draw (v4) -- (v5);
\end{tikzpicture}
\caption{\(T\)-graph}
\label{fig:tgraph}
\end{figure}
In Figure \ref{fig:tgraph} observe that \(\delta(v_1 v_2) = \{v_2 v_4, v_2 v_3\}\) and \(\delta(v_4 v_5) = \{v_2 v_4\}\).

\begin{lem}
\(\displaystyle \phi(2)\cdot 2! = \abs{E}^2 - \abs{E} - \sum_{e \in E} \abs{\delta(e)}\)
\end{lem}
\begin{proof}
    For any edge \(e\) in \(\Gamma\) there exists \(\abs{E} - \abs{\delta(e)} - 1\)
    edges in \(\Gamma\) disjoint from \(e\).
    So by rule of sum
    \[
    \phi(2)\cdot 2! = \sum_{e \in E} \left(\abs{E} - \abs{\delta(e)} - 1\right) = \abs{E}^2 - \abs{E} - \sum_{e \in E} \abs{\delta(e)}.
    \]
\end{proof}

\begin{lem}
    \[
\phi(3)\cdot 3! = \abs{E}^3 - \abs{E}^2 + 2\abs{E} + (-3\abs{E} + 4)\sum_{e \in E}\abs{\delta(e)} + \sum_{e \in E}\abs{\delta(e)}^2 - \sum_{e \in E}\sum_{e' \in \delta(e)} \abs{\delta(e')}
    \]
\end{lem}
\begin{proof}
    \begin{align*}
     & \sum_{e \in E}\left(\sum_{e' \in E \setminus (\delta(e) \cup \{e\})} \abs{E} - \abs{\delta(e)} - \abs{\delta(e')} - 2\right) \\
    =& \sum_{e \in E}\left(\sum_{e' \in E} (\cdots) - \sum_{e' \in \delta(e)} (\cdots) - (\cdots)\right) \\ 
    =& \sum_{e \in E}\abs{E}(\abs{E} - \abs{\delta(e)} - 2) - \sum_{e' \in E} \abs{\delta(e')} - \abs{\delta(e)}(\abs{E} - \abs{\delta(e)} - 2) - \sum_{e' \in \delta(e')}\abs{\delta(e')} - \abs{E} + 2\abs{\delta(e)} + 2 \\
    =& \sum_{e \in E}\abs{E}^2 - \abs{E} + 2 - \sum_{e' \in E} \abs{\delta(e')} + (-2\abs{E} + 4) \abs{\delta(e)} + \abs{\delta(e)}^2 - \sum_{e' \in \delta(e)} \abs{\delta(e')} \\
    =& \abs{E}(\abs{E}^2 - \abs{E} + 2) - \abs{E} \sum_{e \in E}\abs{\delta(e)} +(-2\abs{E} + 4) \sum_{e \in E}\abs{\delta(e)} - \sum_{e \in E}\sum_{e' \in \delta(e)} \abs{\delta(e')} \\
    =& \abs{E}^3 - \abs{E}^2 + 2\abs{E} + (-3\abs{E} + 4)\sum_{e \in E}\abs{\delta(e)} + \sum_{e \in E}\abs{\delta(e)}^2 - \sum_{e \in E}\sum_{e' \in \delta(e)} \abs{\delta(e')}
    \end{align*}

\end{proof}

\begin{lem}
\[
\phi(k)\cdot k! = \sum_{e_0 \in E} \sum_{e_1 \in E_1} \cdots \sum_{e_{k-1} \in E_{k-1}} \left(\abs{E} - k + 1 - \sum_{j=0}^{k} \abs{\delta(e_j)}\right) 
\]
where \(E_j = E \setminus (\{e_0, \cdots, e_{j-1}\} \cup \delta(e_0) \cup \cdots \cup \delta(e_{j-1}))\)
\end{lem}

\begin{figure}[h!]
    \centering
\begin{tikzpicture}[scale=1]
    % Define the two main suspension vertices, positioned closer together
    \node (a1) at (0, 1) {\(a_1\)};
    \node (a2) at (0, -1) {\(a_2\)};

    % Define the intermediate vertices for the 'm' paths
    \node (b1) at (-1, 0) {\(b_1\)};
    \node (bm) at (1, 0) {\(b_m\)};
    \node at (0,0) {\(\dots\)};

    % Draw the edges for the 'm' paths, passing through the b_i nodes
    \draw (a1) to[out=180, in=90] (b1);
    \draw (b1) to[out=270, in=180] (a2);

    \draw (a1) to[out=0, in=90] (bm);
    \draw (bm) to[out=270, in=0] (a2);

    % Define vertices for the 'n1' rays attached to a1, spread out as in the drawing
    \node (c1) at (-1, 2) {\(c_1\)};
    \node (c1p) at (-2, 3) {\(c'_1\)};
    \node (cn1) at (1, 2) {\(c_{n_1}\)};
    \node (cn1p) at (2, 3) {\(c'_{n_1}\)};
    \node at (0, 2.5) {\(\dots\)};

    % Draw edges for the 'n1' rays
    \draw (a1) -- (c1);
    \draw (c1) -- (c1p);
    \draw (a1) -- (cn1);
    \draw (cn1) -- (cn1p);

    % Define vertices for the 'n2' rays attached to a2, also spread out
    \node (d1) at (-1, -2) {\(d_1\)};
    \node (d1p) at (-2, -3) {\(d'_1\)};
    \node (dn2) at (1, -2) {\(d_{n_2}\)};
    \node (dn2p) at (2, -3) {\(d'_{n_2}\)};
    \node at (0, -2.5) {\(\dots\)};

    % Draw edges for the 'n2' rays
    \draw (a2) -- (d1);
    \draw (d1) -- (d1p);
    \draw (a2) -- (dn2);
    \draw (dn2) -- (dn2p);
\end{tikzpicture}
\label{fig:pulsargraph}
\caption{The Pulsar graph \(\mathcal{P}_{m,n_1,n_2}\)}
\end{figure}

Here is the euler characteristic of \(\DUConf_3(\mathcal{P}_{m,n_1,n_2})\).
\begin{align*}
    \frac{19 m^{3}}{6} + 6 m^{2} n_{1} + 6 m^{2} n_{2} - \frac{9 m^{2}}{2} + 6 m n_{1}^{2} + 8 m n_{1} n_{2} - 5 m n_{1} + 6 m n_{2}^{2} - 5 m n_{2}\\
     - \frac{5 m}{3} + \frac{7 n_{1}^{3}}{3} + 5 n_{1}^{2} n_{2} - 3 n_{1}^{2} + 5 n_{1} n_{2}^{2} - 2 n_{1} n_{2} - \frac{10 n_{1}}{3} + \frac{7 n_{2}^{3}}{3} - 3 n_{2}^{2} - \frac{10 n_{2}}{3}
\end{align*}

Here is the difference of it and \(\DUConf_3(\Theta_m)\).
\begin{align*}
3 m^{3} + 6 m^{2} n_{1} + 6 m^{2} n_{2} - 3 m^{2} + 6 m n_{1}^{2} + 8 m n_{1} n_{2} - 5 m n_{1} + 6 m n_{2}^{2} - 5 m n_{2} - 4 m + \frac{7 n_{1}^{3}}{3} \\
+ 5 n_{1}^{2} n_{2} - 3 n_{1}^{2} + 5 n_{1} n_{2}^{2} - 2 n_{1} n_{2} - \frac{10 n_{1}}{3} + \frac{7 n_{2}^{3}}{3} - 3 n_{2}^{2} - \frac{10 n_{2}}{3}
\end{align*}

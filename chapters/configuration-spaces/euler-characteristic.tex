\section{Computing the Euler characteristic of graph configuration spaces}
To compute the euler characteristic of an unorderd \(n\) point graph configuration space we need to count
for each \(0 \le k \le n\) the number of \(k\)-cubes present.
Without further mention \(E = E(\Gamma)\) and \(V = V(\Gamma)\).

A \(k\) cube exists if and only if \(k\) particles are free to move simultaneously while \(n - k\)
particles remain fixed.
Hence the general formula for the number of \(k\) cubes in \(\DUConf_n(\Gamma)\) is
\[
    \abs{\phi(k)} \cdot \begin{pmatrix}\abs{V} - 2k \\ n - k\end{pmatrix}
\]
where \(\phi(k)\) defined as follows.

\begin{defn}
\(\phi(k)\) is the collection consisting of all unordered sets of exactly \(k\) disjoint edges in \(\Gamma\)
where \(\phi(0) = \{\emptyset\}\).
\end{defn}

Note that we define \(\phi(0) = \{\emptyset\}\) so that \(\abs{\phi(0)} = 1\).
With this definition, \(\abs{\phi(k)}\) is also known as the number of ``\(k\)-matchings'' in \(\Gamma\).

Counting the number of \(k\)-matchings for small graphs can be done efficiently 
with a simple sweep over all \(k\) combinations of edges.
However, for general larger graphs this computation quickly becomes unwieldly.

% TODO read/cite Mark Jerrum's paper "Two-dimensional monomer-dimer systems are computationally intractable"
% where he proves computing the matching polynomial for planar graphs is P# ?
% TODO: reference program actually used to make computations? It's trivial especially now with AI hah

\begin{figure}[h!]
\centering
\begin{tabular}{c | c | c}
   \(\Gamma\) & \(n\) & \(\chi(\DUConf_n(\Gamma))\) \\
   \hline
   \(K_5\) & 2 & -5 \\
   \(K_{3,3}\) & 2 & -3 \\
   \(K_5\) & 3 & -5 \\
   \(\Theta_4\) & 3 & -4 \\
   \(K_{3,3}\) & 4 & -3
\end{tabular}
\label{fig:euler_characteristics}
\caption{Euler characteristic of certain unordered graph configuration spaces.}
\end{figure}

A workaround for this computational challenge is determing the Euler characteristic of certain classes of graphs.
In \cite{appiah2024algebraicstructurehyperbolicgraph} the ``pulsar'' graph was defined.


\begin{figure}[h!]
    \centering
\begin{tikzpicture}[scale=1]
    % Define the two main suspension vertices, positioned closer together
    \node (a1) at (0, 1) {\(a_1\)};
    \node (a2) at (0, -1) {\(a_2\)};

    % Define the intermediate vertices for the 'm' paths
    \node (b1) at (-1, 0) {\(b_1\)};
    \node (bm) at (1, 0) {\(b_m\)};
    \node at (0,0) {\(\dots\)};

    % Draw the edges for the 'm' paths, passing through the b_i nodes
    \draw (a1) to[out=180, in=90] (b1);
    \draw (b1) to[out=270, in=180] (a2);

    \draw (a1) to[out=0, in=90] (bm);
    \draw (bm) to[out=270, in=0] (a2);

    % Define vertices for the 'n1' rays attached to a1, spread out as in the drawing
    \node (c1) at (-1, 2) {\(c_1\)};
    \node (c1p) at (-2, 3) {\(c'_1\)};
    \node (cn1) at (1, 2) {\(c_{n_1}\)};
    \node (cn1p) at (2, 3) {\(c'_{n_1}\)};
    \node at (0, 2.5) {\(\dots\)};

    % Draw edges for the 'n1' rays
    \draw (a1) -- (c1);
    \draw (c1) -- (c1p);
    \draw (a1) -- (cn1);
    \draw (cn1) -- (cn1p);

    % Define vertices for the 'n2' rays attached to a2, also spread out
    \node (d1) at (-1, -2) {\(d_1\)};
    \node (d1p) at (-2, -3) {\(d'_1\)};
    \node (dn2) at (1, -2) {\(d_{n_2}\)};
    \node (dn2p) at (2, -3) {\(d'_{n_2}\)};
    \node at (0, -2.5) {\(\dots\)};

    % Draw edges for the 'n2' rays
    \draw (a2) -- (d1);
    \draw (d1) -- (d1p);
    \draw (a2) -- (dn2);
    \draw (dn2) -- (dn2p);
\end{tikzpicture}
\label{fig:pulsargraph}
\caption{The Pulsar graph \(\mathcal{P}_{m,n_1,n_2}\)}
\end{figure}

\begin{thm}
    \label{thm:pulsargrapheuler}
    The Euler characteristic of \(\DUConf_{3}(\mathcal{P}_{m,n_1,n_2})\) is
    \[
    \frac{m^{3}}{6} - \frac{3 m^{2}}{2} - \frac{m n_{1}^{2}}{2} - \frac{3 m n_{1}}{2} - \frac{m n_{2}^{2}}{2} - \frac{3 m n_{2}}{2} + \frac{7 m}{3} - \frac{n_{1}^{3}}{3} + \frac{7 n_{1}}{3} - \frac{n_{2}^{3}}{3} + \frac{7 n_{2}}{3}.
    \]
\end{thm}
\begin{proof}
%TODO: all we really have to do is explain the argument for the 2 and 3 matchings.
\end{proof}

\begin{rem}
Substituting in \(n_1 = 0\) and \(n_2 = 0\) the pular graph becomes \(\Theta_m\).
So, 
the euler characteristic of \(\DUConf_3(\Theta_m)\) is
\[
\frac{m^{3}}{6} - \frac{3 m^{2}}{2} + \frac{7 m}{3}
\]
This aligns with results in section 4 of \cite{appiah2024algebraicstructurehyperbolicgraph}.
\end{rem}
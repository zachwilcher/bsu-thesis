\section{Computing the Euler characteristic of graph configuration spaces}
To compute the euler characteristic of an unorderd \(n\) point graph configuration space we need to count
for each \(0 \le k \le n\) the number of \(k\)-cubes present.

A \(k\) cube exists if and only if \(k\) particles are free to move simultaneously while \(n - k\)
particles remain fixed.
Hence the general formula for the number of \(k\) cubes in \(\DUConf_n(\Gamma)\) is
\[
    \abs{\phi(k)} \cdot\binom{\abs{V(\Gamma)} - 2k}{n - k}
\]
where \(\phi(k)\) defined as follows.

\begin{defn}
\(\phi(k)\) is the collection consisting of all unordered sets of exactly \(k\) disjoint edges in \(\Gamma\)
where \(\phi(0) = \{\emptyset\}\).
\end{defn}

Note that we define \(\phi(0) = \{\emptyset\}\) so that \(\abs{\phi(0)} = 1\).
With this definition, \(\abs{\phi(k)}\) is also known as the number of ``\(k\)-matchings'' in \(\Gamma\).

We now express the Euler characteristic of \(\DUConf_n(\Gamma)\) in the following lemma.
\begin{lem}
\label{lem:eulercharacteristic}
\[
\chi(\DUConf_n(\Gamma)) = \sum_{k=0}^{n} (-1)^k \abs{\phi(k)} \binom{\abs{V(\Gamma)} - 2k}{n - k}
\]
\end{lem}

Counting the number of \(k\)-matchings for small graphs can be done efficiently 
with a simple sweep over all \(k\) combinations of edges.
We have done this for several small graphs (see Figure \ref{fig:euler_characteristics}) that are relevant in later chapters.
\begin{figure}[h!]
\centering
\begin{tabular}{c | c | c}
   \(\Gamma\) & \(n\) & \(\chi(\DUConf_n(\Gamma))\) \\
   \hline
   \(K_5\) & 2 & -5 \\
   \(K_{3,3}\) & 2 & -3 \\
   \(K_5\) & 3 & -5 \\
   \(\Theta_4\) & 3 & -4 \\
   \(K_{3,3}\) & 4 & -3
\end{tabular}
\caption{Euler characteristic of certain unordered graph configuration spaces.}
\label{fig:euler_characteristics}
\end{figure}
% TODO add a link to the code that computes these values.

% TODO read/cite Mark Jerrum's paper "Two-dimensional monomer-dimer systems are computationally intractable"
% where he proves computing the matching polynomial for planar graphs is P# ?
In general, for larger graphs this computation quickly becomes unwieldly.
A workaround for this computational challenge is determining the Euler characteristic of certain classes of graphs.
In \cite{appiah2024algebraicstructurehyperbolicgraph} the \textit{pulsar graph} was defined.
See Figure \ref{fig:pulsargraph} for a depiction of the pulsar graph \(\mathcal{P}_{m,n_1,n_2}\).

\begin{figure}[h!]
    \centering
\begin{tikzpicture}[scale=1]
    % Define the two main suspension vertices, positioned closer together
    \node (a1) at (0, 1) {\(a_1\)};
    \node (a2) at (0, -1) {\(a_2\)};

    % Define the intermediate vertices for the 'm' paths
    \node (b1) at (-1, 0) {\(b_1\)};
    \node (bm) at (1, 0) {\(b_m\)};
    \node at (0,0) {\(\dots\)};

    % Draw the edges for the 'm' paths, passing through the b_i nodes
    \draw (a1) to[out=180, in=90] (b1);
    \draw (b1) to[out=270, in=180] (a2);

    \draw (a1) to[out=0, in=90] (bm);
    \draw (bm) to[out=270, in=0] (a2);

    % Define vertices for the 'n1' rays attached to a1, spread out as in the drawing
    \node (c1) at (-1, 2) {\(c_1\)};
    \node (c1p) at (-2, 3) {\(c'_1\)};
    \node (cn1) at (1, 2) {\(c_{n_1}\)};
    \node (cn1p) at (2, 3) {\(c'_{n_1}\)};
    \node at (0, 2.5) {\(\dots\)};

    % Draw edges for the 'n1' rays
    \draw (a1) -- (c1);
    \draw (c1) -- (c1p);
    \draw (a1) -- (cn1);
    \draw (cn1) -- (cn1p);

    % Define vertices for the 'n2' rays attached to a2, also spread out
    \node (d1) at (-1, -2) {\(d_1\)};
    \node (d1p) at (-2, -3) {\(d'_1\)};
    \node (dn2) at (1, -2) {\(d_{n_2}\)};
    \node (dn2p) at (2, -3) {\(d'_{n_2}\)};
    \node at (0, -2.5) {\(\dots\)};

    % Draw edges for the 'n2' rays
    \draw (a2) -- (d1);
    \draw (d1) -- (d1p);
    \draw (a2) -- (dn2);
    \draw (dn2) -- (dn2p);
\end{tikzpicture}
\caption{The Pulsar graph \(\mathcal{P}_{m,n_1,n_2}\)}
\label{fig:pulsargraph}
\end{figure}

\begin{thm}
    \label{thm:pulsargrapheuler}
    The Euler characteristic of \(\DUConf_{3}(\mathcal{P}_{m,n_1,n_2})\) is
    \[
    \frac{m^{3}}{6} - \frac{3 m^{2}}{2} - \frac{m n_{1}^{2}}{2} - \frac{3 m n_{1}}{2} - \frac{m n_{2}^{2}}{2} - \frac{3 m n_{2}}{2} + \frac{7 m}{3} - \frac{n_{1}^{3}}{3} + \frac{7 n_{1}}{3} - \frac{n_{2}^{3}}{3} + \frac{7 n_{2}}{3}.
    \]
\end{thm}
\begin{proof}
Since there are no \(k\)-cubes when \(k > 3\), ``all'' we need to do is compute the number of \(k\)-matchings
for \(k = 0, 1, 2, 3\).
We introduce some terminology to make the counting easier.
An ``outer emission edge'' is one of the edges \(c_i c_i'\) or \(d_i d_i'\).
An ``inner emission edge'' is one of the edges \(a_1 c_i\) or \(a_2 d_i\).
A ``middle edge'' is one of the edges \(a_1 b_i\) or \(a_2 b_i\).
We then count the number of matching by considering edges taken from each of these three categories.


\textbf{3-matchings}
The following table summarizes the computations for each case.

\begin{center}
\begin{tabular}{c | c | c | c | c }
Subcase & Outer & Inner & Middle & Count \\ 
\hline
1 & 3 & 0 & 0 & \(\binom{n_1 + n_2}{3}\) \\
2 & 2 & 1 & 0 & \((n_1 + n_2)\binom{n_1 + n_2 - 1}{2}\) \\
3 & 2 & 0 & 1 & \(2m\binom{n_1 + n_2}{2}\) \\
4 & 1 & 2 & 0 & \(n_1n_2(n_1 + n_2 - 2)\) \\
5 & 1 & 1 & 1 & \(m n_2 (n_1 + n_2 - 1) + m n_1 (n_1 + n_2 - 1)\) \\
6 & 1 & 0 & 2 & \(m(m-1)(n_1 + n_2)\) \\
7 & 0 & 3 & 0 & 0 \\
8 & 0 & 2 & 1 & 0 \\
9 & 0 & 1 & 2 & 0 \\
10 & 0 & 0 & 3 & 0 
\end{tabular}
\end{center}

We elaborate on each of these cases below.
\begin{enumerate}
    \item All three edges are chosen from the outer emission edges.
    There are \(\binom{n_1 + n_2}{3}\) such matchings.

    \item One edge is chosen from the \(n_1 + n_2\) inner emission edges,
    and two edges are chosen from the outer emission edges. Since there are \(n_1 + n_2 - 1\)
    outer emission edges disjoint from the chosen inner emission edge,
    there are \((n_1 + n_2)\binom{n_1 + n_2 - 1}{2}\) total such matchings.

    \item One edge is chosen from the \(2m\) middle edges, and two edges are chosen from the outer emission edges.
    Since there are \(n_1 + n_2\) outer emission edges disjoint from the chosen middle edge,
    there are \( 2m\binom{n_1 + n_2}{2} \) such matchings.

    \item Two edges are chosen from the \(n_1 + n_2\) inner emission edges, and one edge is chosen from the outer emission edges.
    Since the only way to chose two disjoint inner emission edges is to choose one from the \(n_1\) edges
    attached to \(a_1\) and one from the \(n_2\) edges attached to \(a_2\),
    there are \(n_1 n_2\) such choices.
    After choosing two inner emission edges, there are \(n_1 + n_2 - 2\) outer emission edges disjoint from the chosen inner emission edges.


    \item One edge is chosen from a middle edge, one edge is chosen from an inner emission edge, and one edge is chosen from an outer emission edge.
    If we pick one of the \(n_1\) top inner emission edges, there are \(m\) middle edges disjoint from it, and \((n_1 + n_2 - 1)\)
    outer emission edges disjoint from both the chosen inner emission edge and middle edge.
    Similarly, if we pick one of the \(n_2\) bottom inner emission edges,
    there are \(m\) middle edges disjoint from it, and \((n_1 + n_2 - 1)\)
    outer emission edges disjoint from both the chosen inner emission edge and middle edge.


    \item Two edges are chosen from the middle edges, and one edge is chosen from the outer emission edges.
    After choosing one of the top \(m\) middle edges, there are \(m - 1\) disjoint middle edges remaining.
    This accounts for all ways to choose to disjoint pairs of middle edges.
    Since there are \(n_1 + n_2\) outer emission edges disjoint from any middle edge,
    there are \(m (m - 1) (n_1 + n_2)\) such matchings.
\end{enumerate}

\textbf{2-matchings:}
The following table summarizes the computations for each case.
\begin{center}
\begin{tabular}{c | c | c | c | c }
Subcase & Outer & Inner & Middle & Count \\
\hline
1 & 2 & 0 & 0 & \(\binom{n_1 + n_2}{2}\) \\
2 & 1 & 1 & 0 & \( (n_1 + n_2) (n_1 + n_2 - 1) \) \\
3 & 1 & 0 & 1 & \( 2m (n_1 + n_2) \) \\
4 & 0 & 2 & 0 & \( n_1 n_2 \) \\
5 & 0 & 1 & 1 & \( m n_2 + m n_1 \) \\
6 & 0 & 0 & 2 & \( m (m - 1)\)
\end{tabular}
\end{center}
Again, we elaborate on each of these cases below.
\begin{enumerate}
    \item Both edges are chosen from the outer emission edges. There are \(\binom{n_1 + n_2}{2}\) such matchings.
    \item Choose one of the \((n_1 + n_2)\) inner emission edges, then there are \((n_1 + n_2 - 1)\) possible disjoint outer emission edges.
    \item Choose one of the \(2m\) middle edges, then there are \((n_1 + n_2)\) possible disjoint outer emission edges.
    \item Choose one of the \(n_1\) top inner emission edges, then there are \(n_2\) bottom possible disjoint inner emission edges.
    \item If we choose one of the \(m\) top middle edges, then there are \(n_2\) possible bottom disjoint inner emission edges.
    Similarly, if we choose one of the \(m\) bottom middle edges, then there are \(n_1\) possible top disjoint inner emission edges.
    \item Choose one of the \(m\) top middle edges, then there are \((m-1)\) possible disjoint bottom middle edges.
\end{enumerate}

\textbf{1-matchings:}
This case is much more straightforward. We simply add up the number of ways to select one edge from each of the three categories.
Hence the total number of 1-matchings is
\[
(n_1 + n_2) + (n_1 + n_2) + 2m
\]

\textbf{0-matchings:}
There is exactly one 0-matching, the empty set.


\textbf{Final Count}
Now that we know the number of \(k\)-matchings for \(k = 0, 1, 2, 3\), we can compute the Euler characteristic
with the formula from Lemma \ref{lem:eulercharacteristic}.
The simplification was not done by hand, but rather with the help of a computer algebra system namely SymPy.
% TODO add a link to the code used.
\end{proof}

\begin{rem}
Substituting in \(n_1 = 0\) and \(n_2 = 0\) the pular graph becomes \(\Theta_m\) aka. \(K_{2,m}\).
So, 
the euler characteristic of \(\DUConf_3(\Theta_m)\) is
\[
\frac{m^{3}}{6} - \frac{3 m^{2}}{2} + \frac{7 m}{3}
\]
This aligns with results in section 4 of \cite{appiah2024algebraicstructurehyperbolicgraph}.
\end{rem}
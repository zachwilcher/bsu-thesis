\section{A Long List of Lemmas}
From this point forward in the section we assume that \(\DConf_n(\Gamma)\) is an \(m\)-manifold without boundary
and \(M\) is an \((m-1)\)-matching in \(\Gamma\).

\subsection{Minimum Graph Size and Key Lemma}

\begin{lem}
\label{lem:graph-size}
    \(\Gamma\) has at least \(m + n\) vertices.
\end{lem}


\begin{lem}
    \label{lem:special-edges}
    For any collection of \(n - (m - 1)\) vertices in \(\Gamma - V(M)\), there exists exactly two
    edges \(e_1\) and \(e_2\) such that
    ...
\end{lem}

\subsection{\(m = 1\) or No Cycles}

\begin{lem}
    Every vertex in \(\Gamma - V(M)\) has degree at most \(3\).
\end{lem}

\begin{lem}
    If \(\Gamma - V(M)\) contains a degree \(3\) vertex,
    then \(\Gamma - V(M) \cong K_{1,3}\) and \(n = m + 1\).
\end{lem}

\begin{lem}
    If \(m = 1\), then \(\Gamma \cong K_{1,3}\) and \(n = 2\),
    or \(\Gamma\) is one or more cycles and \(n \in \{1, \abs{V(\Gamma)} - 1\}\) 
\end{lem}

\begin{proof}
    It is sufficient to show that the degree of every vertex in \(\Gamma\) is exactly \(2\).
    Let \(v\) be some vertex in \(\Gamma\). By Lemma \ref{lem:graph-size}, there exists \(n\) other vertices in \(\Gamma\).
    The edges \(e_1\) and \(e_2\) guaranteed by Lemma \ref{lem:special-edges} must then both be incident to \(v\)
    and there can be no other edges incident to \(v\) and these \(n\) vertices.
    If there were another edge incident to \(v\), then \(v\) would have degree at least \(3\)
    meaning \(\Gamma\) would have to be \(K_{1,3}\).
    So, then \(n = 2\) and \(\Gamma\) is just \(K_{1,3}\).

    Otherwise, \(v\) has degree exactly \(2\).
    Since \(v\) was arbitrary, every vertex in \(\Gamma\) has degree exactly \(2\)
    meaning, \(\Gamma\) is one or more cycles.
    If \(1 < n < \abs{V(\Gamma)} - 1\), then there are too many ways for particles to move.
\end{proof}

Now that the \(m = 1\) case is completely determined, we only consider when \(m \ge 2\) from this point onward.

\begin{lem}
    \(\Gamma - V(M)\) must be \(K_{1,3}\) or contains a cycle.
\end{lem}

\begin{proof}
    Suppose that \(\Gamma - V(M)\) contains no vertices of degree 3 nor any cycles.
    Then \(\Gamma - V(M)\) is a forest whose trees are paths.
    Since \(\Gamma - V(M)\) has at least \(m + n - 2(m - 1) = n - m + 2\) vertices by Lemma \ref{lem:graph-size},
    we can place \(n - (m - 1)\) particles filling up the paths.
    Then there is only one edge contradicting Lemma \ref{lem:special-edges}.
\end{proof}

\subsection{\(m > 1\) and Cycles}
Since we know what happens when \(m = 1\) or \(\Gamma - V(M)\) contains no cycles.
Continue to assume that \(m > 1\) but now also suppose \(\Gamma - V(M)\) contains a cycle \(C\).

\begin{lem}
    \label{lem:cycle-length-1}
    If \(n > m\), then \(C\) has length at most \(n - m + 2\).
\end{lem}

\begin{proof}
    By Lemma \ref{lem:graph-size} \(\Gamma - V(M)\) has at least \(n + m - 2(m - 1) = n - m + 2\) vertices.
    So, after picking any \(n - (m - 1) = n - m + 1\) vertices, there must be at least one other vertex in \(\Gamma - V(M)\).

    Suppose \(C\) had more than \(n - m + 2\) vertices.
    Since \(n > m\), it follows that \(n - m + 2 \ge 3\) and that \(C\) has at least \(4\) vertices.
    So, after picking \(n - m + 1\) vertices in a path on \(C\), there are still at least \(2\) vertices left on \(C\)
    contradicting Lemma \ref{lem:special-edges}.
\end{proof}

\begin{lem}
    If \(n > m\) and \(C\) has length \(n - m + 2\),
    then \(\Gamma - V(M) = C\).
\end{lem}

\begin{proof}
    Suppose there was another vertex \(v\) in \(\Gamma - V(M) - V(C)\).
    Then, after picking this vertex and \(n - m\) vertices on \(C\),
    there are two vertices left on \(C\) contradicting Lemma \ref{lem:special-edges}.
\end{proof}

\begin{lem}
    If \(n > m\), then \(n = \abs{V(M)}/2 + \abs{V(C)} - 1\)
\end{lem}

\begin{lem}
    If \(n > m\), then \(\Gamma - V(M)\) has at most \(4\) vertices.
\end{lem}

\begin{proof}
We proceed by cases on the length of \(C\).

\textbf{Case 1:} \(C\) has length \(3\)

\textbf{Case 2:} \(C\) has length \(4\)

\textbf{Case 3:} \(C\) has length greater than \(5\)
\end{proof}

\begin{lem}
If \(n > m\) and \(C\) has length \(3\), then \(\Gamma - V(M) \cong K_3\) or \(\Gamma - V(M) \cong K_3 \cup K_1\).
\end{lem}

\begin{lem}
If \(n > m\) and \(C\) has length \(4\), then \(\Gamma - V(M) \cong K_{2,2}\)
\end{lem}

\begin{lem}
If \(n = m\), then \(C\) has length at most \(4\).
\end{lem}

\begin{lem}
If \(n = m\) and \(C\) has length \(3\), then \(\Gamma - V(M) \cong K_3\).
\end{lem}

\begin{lem}
If \(n = m\) and \(C\) has length \(4\), then \(\Gamma - V(M) \cong K_{2,2}\).
\end{lem}



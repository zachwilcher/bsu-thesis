\section{Exclusivity of Subgraphs}

Now that we have determined the possible subgraphs \(\Gamma - V(M)\) for some \((m-1)\)-matching \(M\),
we claim that if \(\Gamma - V(M_0) \cong \Lambda\) for some \((m-1)\)-matching \(M_0\) and subgraph \(\Lambda\),
then \(\Gamma - V(M) \cong \Lambda\) for any \((m-1)\)-matching \(M\).

\begin{thm}
    Suppose \(\Conf_n(\Gamma)\) is an \(m\)-manifold without boundary for \(m \ge 2\),
    then there exists a subgraph \(\Lambda\) in \(\Gamma\)
    such that \(\Lambda \in \{K_3, K_{2,2}, K_3 \cup K_1, K_{1,3}\}\)
    and \(\Gamma - V(M) \cong \Lambda\) for any \((m-1)\)-matching \(M\) in \(\Gamma\).
\end{thm}

\begin{proof}
If \(\Lambda\) is \(K_{2,2}\) then \(n\) can be \(m\) or \(m + 2\).
In either case we cannot have that \(\Gamma - V(M) \cong K_{1,3}\) or \(\Gamma - V(M) \cong K_3 \cup K_1\)
since \(n\) must be \(m + 1\) there.
Furthermore, we cannot have that \(\Gamma - V(M) \cong K_3\) while \(\Lambda \cong K_{2,2}\)
since that would imply the number of vertices in \(\Gamma\) is different in each case.
So, \(K_{2,2}\) cannot occur simultaneously with any of the other subgraphs.

Similarly, \(K_3\) cannot occur simultaneously with \(K_{1,3}\) or \(K_3 \cup K_1\)
since the number of vertices in \(\Gamma\) would total up to different amounts.

So, we just need to show that \(K_{1,3}\) and \(K_3 \cup K_1\) cannot occur simultaneously.
% TODO 

\end{proof}

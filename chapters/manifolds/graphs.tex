\section{Graphs}
Let \(m \ge 1\) and suppose \(\Gamma\) contains an \(m\)-matching i.e. \(\Gamma\) has at least \(m\) disjoint edges.


\begin{lem}
Let \(n \ge 3\). If \(\Gamma - V(M) \cong K_n\) for any \(m\)-matching \(M\), then \(\Gamma \cong K_{2m+n}\).
\end{lem}

\begin{proof}
    Suppose that \(\Gamma - V(M) \cong K_n\) for any \(m\)-matching \(M\).
    Notice that \(\abs{V(\Gamma)} = \abs{V(M)} + \abs{V(K_n)} = 2m + n\).
    Let \(v_1\) and \(v_2\) be any two distinct vertices in \(\Gamma\) and \(M\) be an \(m\)-matching.
    If \(v_1\) and \(v_2\) belong to \(\Gamma - M\), then they're adjacent since they belong to a complete graph.
    So, suppose that \(v_1\) or \(v_2\) belongs to \(M\).
    Without loss of generality, suppose \(v_1\) belong to \(M\) and let \(w_1\) be the vertex adjacent to \(v_1\) in \(M\).

    If \(v_2\) belongs to \(M\), then let \(w_2\) be the vertex adjacent to \(v_2\) in \(M\).
    Let \(u_1, u_2\) be two adjacent vertices in \(\Gamma - M\) and define \(M' = (M \cup \{u_1, u_2\}) \setminus \{v_2, w_2\}\).
    Then, \(M'\) is an \(m\)-matching containing \(v_1\) but does not contain \(v_2\).
    So, we can assume without a loss of generality that \(v_1\) belongs to \(M\) but \(v_2\) does not.

    Since \(v_2\) does not belong to \(M\) and \(n \ge 3\), there are at least two adjacent vertices \(u_1\) and \(u_2\) other than \(v_2\) in \(\Gamma - M\).
    Let \(M' = (M \cup \{u_1, u_2\}) \setminus \{v_1, w_1\}\).
    Since \(M'\) is an \(m\)-matching, \(v_1\) and \(v_2\) belong to the complete graph \(\Gamma - M'\) and must be adjacent.
    Since \(v_1\) and \(v_2\) were arbitrary, it follows that \(\Gamma\) is complete.
\end{proof}

\begin{lem}
If \(\Gamma - V(M) \cong C_4\) for any \(m\)-matching \(M\), then \(\Gamma \cong K_{m+2, m+2}\).
\end{lem}

\begin{proof}
\end{proof}


\begin{lem}
    If \(\Gamma - V(M) \cong K_{1,3}\) for any \(m\)-matching \(M\), then \(\Gamma \cong K_{m+1, m+3}\).
\end{lem}

\begin{proof}
    Suppose that \(\Gamma - V(M) \cong K_{1,3}\) for any \(m\)-matching \(M\).
    Notice that \(\abs{V(\Gamma)} = \abs{V(M)} + \abs{V(K_{1,3})} = 2m + 4\).
    Let \(M_0\) be an \(m\)-matching, \(a_{m+1}, a_{m+2}, a_{m+3}\) be the valence \(1\) vertices in \(\Gamma - V(M_0)\)
    and \(b_{m+1}\) be the valence \(3\) vertex in \(\Gamma - V(M_0)\).

    Label the \(m\) disjoint edges in \(M_0\) as \(v_1 w_1, v_2 w_2, \cdots, v_m w_m\).
    Let \(M_i = (M_0 \cup \{a_{m+1}b_{m+1}\}) \setminus \{ v_i w_i \}\) for \(1 \le i \le m\).
    Then, \(v_i\) and \(w_i\) belong to \(\Gamma - V(M_i)\).
    Since \(v_i\) and \(w_i\) are adjacent and \(\Gamma - V(M_i) \cong K_{1,3}\),
    one of these two vertices has valence \(1\) and the other has valence \(3\) in \(\Gamma - V(M_i)\).
    Let \(a_i\) be the valence \(1\) vertex and \(b_i\) be the valence \(3\) vertex.
    Note that \(b_i\) must also be adjacent to \(a_{m+2}\) and \(a_{m+3}\).
    Let \(A = \{a_i\}_{i=1}^{m+3}\) and \(B = \{b_j\}_{j=1}^{m+1}\).

    \textbf{Claim 1:} Every vertex in \(A\) is adjacent to every vertex in \(B\) in \(\Gamma\).
    To see this, let \(a_i \in A\) and \(b_j \in B\).
    Define \(M^* = \left(\bigcup_{k=1}^{m+1}\{a_k b_k \} \setminus \{a_ib_i, a_jb_j\}\right)\cup\{a_{m+2}b_i\}\)
    Notice that \(M^*\) has \(m-1\) disjoint edges for each \(a_k b_k\) when \(k \neq i\) and \(k \neq j\)
    and has an \(m\)th disjoint edge \(a_{m+2}b_i\). Therefore, \(M^*\) is an \(m\)-matching, meaning \(\Gamma - V(M^*) \cong K_{1,3}\).
    It remains to show that \(a_i\) and \(b_j\) are adjacent in \(\Gamma - V(M^*)\).
    Since \(\Gamma - V(M^*) \cong K_{1,3}\), if \(a_i\) and \(b_j\) were not adjacent in \(\Gamma - V(M^*)\), 
    then they would both have valence \(1\) in \(\Gamma - V(M^*)\).
    However, \(b_j\) is adjacent to both \(a_j\) and \(a_{m+3}\) in \(\Gamma - V(M^*)\) (from our original construction of \(a_j\) and \(b_j\)), meaning \(b_j\) has valence at least \(2\) in \(\Gamma - V(M^*)\).
    Therefore, \(b_j\) must be the central vertex in \(\Gamma - V(M^*)\), meaning \(a_i\) must be adjacent to \(b_j\).

    \textbf{Claim 2:} No two vertices in \(A\) are adjacent in \(\Gamma\).
    To see this, suppose \(a_i, a_j \in A\) are adjacent in \(\Gamma\) for some \(i \neq j\).
    Then, \(\{a_i, a_j, b_i\}\) form a triangle in \(\Gamma\).
    Let \(a_l \in A\) be distinct from \(a_i\) and \(a_j\). 
    % TODO: This matching is invalid. We need to just find m edges disjoint from ai aj and bi.
    Define \(M^{**} = \left(\bigcup_{k=1}^{m+1}\{a_k b_k\}\setminus\{a_i b_i, a_j b_j\}\right)\cup\{a_l b_j\}\).
    Notice that \(M^{**}\) has \(m-1\) edges for each edge \(a_k b_k\) when \(k \neq i\) and \(k \neq j\)
    and \(m\)th edge \(a_l b_j\).
    Therefore \(M^{**}\) is an \(m\)-matching and  \(\Gamma - V(M^{**}) \cong K_{1,3}\).
    However, \(\{a_i, a_j, b_i\}\) forms a triangle in \(\Gamma - V(M^{**})\), a contradiction.

    \textbf{Claim 3:} No two vertices in \(B\) are adjacent in \(\Gamma\).
    Suppose \(b_i, b_j \in B\) are adjacent in \(\Gamma\) for some \(i \neq j\).
    Then, \(\{b_i, b_j, a_i\}\) form a triangle in \(\Gamma\).
    Define \(M^{***} = \left(\bigcup_{k=1}^{m+1}\{a_k b_k\}\setminus\{a_i b_i, a_j b_j\}\right)\cup\{b_i b_j\}\).
    Notice that \(M^{***}\) has \(m-1\) edges for each edge \(a_k b_k\) when \(k \neq i\) and \(k \neq j\)
    and \(m\)th edge \(b_i b_j\). Therefore \(M^{***}\) is an \(m\)-matching and \(\Gamma - V(M^{***}) \cong K_{1,3}\).
    However, \(\{a_i, a_j, a_{m+2}, a_{m+3}\}\) belong to \(\Gamma - V(M^{***})\).
    Since \(\Gamma - V(M^{***}) \cong K_{1,3}\) is connected, we contradict claim 2.

    So, \(A\) and \(B\) form two partite sets in \(\Gamma\) making \(\Gamma\) bipartite.
    Since every vertex in \(A\) is adjacent to every vertex in \(B\), it follows that \(\Gamma \cong K_{m+1, m+3}\).
\end{proof}

\begin{lem}
Let \(p,q \ge 1\) such that \(p + q \ge 4\). If \(\Gamma - V(M) \cong K_{p, q}\) for any \(m\)-matching \(M\), then \(\Gamma \cong K_{m+p, m+q}\).
\end{lem}

\begin{proof}
    Without a loss of generality, assume \(p \ge q\). Then, \(p + p \ge p + q \ge 4\) meaning, \(p \ge 2\).
    Suppose that \(\Gamma - V(M) \cong K_{p,q}\) for any \(m\)-matching \(M\).
    Notice that \(\abs{V(\Gamma)} = \abs{V(M)} + \abs{V(K_{p,q})} = 2m + p + q\).
    Let \(M_0\) be an \(m\)-matching, \(\{a_{m+i}\}_{i=1}^p\) be the vertices in the partite set of size \(p\) in \(\Gamma - V(M_0)\),
    and \(\{b_{m+j}\}_{j=1}^q\) be the vertices in the partite set of size \(q\) in \(\Gamma - V(M_0)\).
    If \(p = q\), then ensure the the vertices are labeled so that each \(a_{m+i}\) is adjacent to each \(b_{m+j}\).

    Label the \(m\) disjoint edges in \(M_0\) as \(v_1 w_1, v_2 w_2, \cdots, v_m w_m\).
    We define a new \(m\)-matching for each \(1 \le i \le m\).
    \[
         M_i = (M_0 \cup \{a_{m+1}b_{m+1}\}) \setminus \{ v_i w_i \}
    \]
    Then, \(v_i\) and \(w_i\) belong to \(\Gamma - V(M_i)\).
    Since \(v_i\) and \(w_i\) are adjacent and \(\Gamma - V(M_i) \cong K_{p,q}\),
    one of these two vertices belongs to the partite set of size \(p\) and belongs to the partite set of size \(q\) in \(\Gamma - V(M_i)\).
    Let \(a_i\) be the vertex in the partite set of size \(p\) and \(b_i\) be the vertex in the partite set of size \(q\).
    If \(p = q\), then ensure that \(a_i\) is adjacent to each \(b_{m+k}\) for \(1 \le k \le q\)
    \textbf{and} \(b_i\) is adjacent to each \(a_{m+k}\) for \(1 \le k \le p\).

    \textbf{Claim 1:} Every vertex in \(A\) is adjacent to every vertex in \(B\) in \(\Gamma\).
    To see this, let \(a_i \in A\) and \(b_j \in B\). 
    Depending on the values of \(i\) and \(j\), we may already know that \(a_i\) and \(b_j\) are adjacent in \(\Gamma\).
    \begin{enumerate}
        \item If \(i = j\), then \(a_i\) and \(b_j\) are adjacent in \(\Gamma\) by construction.
        \item If \(i \ge m +1\) and \(j \ge m +1\), then \(a_i\) and \(b_j\) are adjacent in \(\Gamma\) since they belong to different partite sets in \(\Gamma - V(M_0)\).
        \item If \(i \ge m + 2\) and \(j \le m\), then \(a_i\) and \(b_j\) belong to different partite sets in \(\Gamma - V(M_j)\) and hence are connected.
        \item If \(i \le m\) and \(j \ge m + 2\), then \(a_i\) and \(b_j\) belong to different partite sets in \(\Gamma - V(M_i)\) and hence are connected.
    \end{enumerate}
    So, suppose \(i\) and \(j\) satisfy none of the above conditions i.e. \(i \neq j\), and \(i \le m + 1\), and \(j \le m + 1\), and either \(i \le m\) or \(j \le m\).

    % TODO continue generalizing from here.
    We now construct a new \(m\)-matching as follows.
    \[
        M' = \left(\bigcup_{k=1}^{m+1}\{a_k b_k \} \setminus \{a_ib_i, a_jb_j\}\right)\cup\{a_j b_i\}
    \]
    Notice that \(M'\) has \(m-1\) disjoint edges for each \(a_k b_k\) when \(k \neq i\) and \(k \neq j\)
    and has an \(m\)th disjoint edge \(a_j b_i\). Therefore, \(M'\) is an \(m\)-matching, meaning \(\Gamma - V(M') \cong K_{p,q}\).
    We will show that \(a_i\) and \(b_j\) are adjacent in \(\Gamma - V(M')\).
    Since \(\Gamma - V(M') \cong K_{p,q}\), if \(a_i\) and \(b_j\) were not adjacent in \(\Gamma - V(M')\), 
    then they would belong to the same partite set in \(\Gamma - V(M')\).

    However, \(b_j\) is adjacent to both \(a_j\) and \(a_{m+3}\) in \(\Gamma - V(M')\) (from our original construction of \(a_j\) and \(b_j\)), meaning \(b_j\) has valence at least \(2\) in \(\Gamma - V(M^*)\).
    Therefore, \(b_j\) must be the central vertex in \(\Gamma - V(M')\), meaning \(a_i\) must be adjacent to \(b_j\).

    \textbf{Claim 2:} No two vertices in \(A\) are adjacent in \(\Gamma\).
    To see this, suppose \(a_i, a_j \in A\) are adjacent in \(\Gamma\) for some \(i \neq j\).
    Then, \(\{a_i, a_j, b_i\}\) form a triangle in \(\Gamma\).
    Let \(a_l \in A\) be distinct from \(a_i\) and \(a_j\). 
    % TODO: This matching is invalid. We need to just find m edges disjoint from ai aj and bi.
    Define \(M^{**} = \left(\bigcup_{k=1}^{m+1}\{a_k b_k\}\setminus\{a_i b_i, a_j b_j\}\right)\cup\{a_l b_j\}\).
    Notice that \(M^{**}\) has \(m-1\) edges for each edge \(a_k b_k\) when \(k \neq i\) and \(k \neq j\)
    and \(m\)th edge \(a_l b_j\).
    Therefore \(M^{**}\) is an \(m\)-matching and  \(\Gamma - V(M^{**}) \cong K_{1,3}\).
    However, \(\{a_i, a_j, b_i\}\) forms a triangle in \(\Gamma - V(M^{**})\), a contradiction.

    \textbf{Claim 3:} No two vertices in \(B\) are adjacent in \(\Gamma\).
    Suppose \(b_i, b_j \in B\) are adjacent in \(\Gamma\) for some \(i \neq j\).
    Then, \(\{b_i, b_j, a_i\}\) form a triangle in \(\Gamma\).
    Define \(M^{***} = \left(\bigcup_{k=1}^{m+1}\{a_k b_k\}\setminus\{a_i b_i, a_j b_j\}\right)\cup\{b_i b_j\}\).
    Notice that \(M^{***}\) has \(m-1\) edges for each edge \(a_k b_k\) when \(k \neq i\) and \(k \neq j\)
    and \(m\)th edge \(b_i b_j\). Therefore \(M^{***}\) is an \(m\)-matching and \(\Gamma - V(M^{***}) \cong K_{1,3}\).
    However, \(\{a_i, a_j, a_{m+2}, a_{m+3}\}\) belong to \(\Gamma - V(M^{***})\).
    Since \(\Gamma - V(M^{***}) \cong K_{1,3}\) is connected, we contradict claim 2.

    So, \(A\) and \(B\) form two partite sets in \(\Gamma\) making \(\Gamma\) bipartite.
    Since every vertex in \(A\) is adjacent to every vertex in \(B\), it follows that \(\Gamma \cong K_{m+1, m+3}\).
\end{proof}
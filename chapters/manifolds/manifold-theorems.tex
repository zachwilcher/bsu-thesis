\section{Theorems}

In an earlier section, we determined what \(\Gamma\) and \(n\) would have to be if \(\Conf_n(\Gamma)\) is a \(1\)-manifold without boundary.
We restate that result here.

\begin{thm}
  Let \(V\) be the number of vertices in \(\Gamma\).
  If \(\DConf_n(\Gamma)\) is a \(1\)-manifold without boundary, then \(\DConf_n(\Gamma)\) is one of the following.
\begin{enumerate}
  \item \(\DConf_1(\sqcup C_k) \cong \DConf_{V - 1}(\sqcup C_k)\) where \(C_k\) is a cycle graph.
  \item \(\DConf_2(K_{1,3})\)
\end{enumerate}
\end{thm}

\begin{thm}
  Let \(m \ge 2\). If \(\DConf_n(\Gamma)\) is an \(m\)-manifold without boundary, 
  then \(\DConf_n(\Gamma)\) is one of the following.
\begin{enumerate}
  \item \(\DConf_m(K_{2m+1}) \cong \DConf_{m+1}(K_{2m+1})\)
  \item \(\DConf_m(K_{m+1, m+1}) \cong \DConf_{m+2}(K_{m+1, m+1})\)
  \item \(\DConf_{m+1}(K_{2m + 1} \cup K_1)\)
  \item \(\DConf_{m+1}(K_{m, m+2})\)
\end{enumerate}
\end{thm}

\begin{proof}
  Let \(M\) be an \((m-2)\) matching in \(\Gamma\).
  As shown in the previous section, \(\Gamma - N(M)\) is a unique subgraph of \(\Gamma\).

  % TODO
\end{proof}

In \cite{abrams2000configurationspaces}, Abrams showed the following.
\begin{prop}
If \(m > 2\) then \(\DConf_m(K_{(m+1), (m+1)})\) and \(\DConf_{m}(K_{(2m+1)})\) are not \(m\)-manifolds.
\end{prop}
Applying his duality theorem, it follows that
\(\DConf_{m+1}(K_{m+1, m+1})\) and \(\DConf_{m+1}(K_{2m+1})\)
are not \(m\) manifolds either.


\begin{prop}
Same problem happens with \(\Gamma = K_{m, m+2}\) or \(\Gamma = K_{2m+1}\cup K_1\) when \(m > 2\).
\end{prop}

\begin{cor}
\(\DConf_n(\Gamma)\) is an \(m\)-manifold without boundary if and only if it is one of the following.
\begin{enumerate}
  \item \(\DConf_1(\sqcup C_k) \cong \DConf_{V - 1}(\sqcup C_k)\) where \(C_k\) is a cycle graph.
  \item \(\DConf_2(K_{1,3})\)
  \item \(\DConf_2(K_{5}) \cong \DConf_3(K_5)\)
  \item \(\DConf_2(K_{3,3}) \cong \DConf_4(K_{3,3})\)
  \item \(\DConf_3(K_{2,4})\)
  \item \(\DConf_3(K_3 \cup K_1)\)
\end{enumerate}
\end{cor}
\begin{proof}
  It remains to show that these spaces are actually manifolds.
  % TODO
\end{proof}

\begin{question}
  What about manifolds with boundary?
\end{question}

One approach to this question is to try and extend the technique of Lemma \ref{lem:special-edges}
to account for boundary points.
A boundary point would have to correspond to a configuration in \(\Gamma - V(M)\)
where only one particle can move. This gives a third possibility for the edges
guaranteed by this lemma.

Alternatively, one may be able to think about the known manifolds and see what happens
when boundary is introduced by cutting out parts of the space.
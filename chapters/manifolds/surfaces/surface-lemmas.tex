\subsection{Key Lemmas}
The example graphs listed in Figure \ref{fig:euler_characteristics} are in fact the only connected graphs
whose discretized configuration spaces are surfaces.

\begin{lem}
\label{lem:is_surface_0}
If \(\DConf_n(\Gamma)\) is an \(m\)-manifold with or without boundary, then \(\Gamma\) has at least \(n+m\) vertices and \(n \ge m\).
\end{lem}
\begin{proof}
    If \(\Gamma\) has less than \(n\) vertices, then \(\DConf_n(\Gamma)\) is empty;
    so, suppose \(\Gamma\) has at least \(n\) vertices.
    If \(\Gamma\) has less than \(n + m\) vertices, then
    for any configuration of \(n\) particles on \(\Gamma\)
    at most \(m-1\) particles can simultaneously move.
    Since exactly \(m\) particles need to be able to simultaneously move for an \(m\)-cube 
    to exist in \(\DConf_n(\Gamma)\), no configuration in \(\DConf_n(\Gamma)\)
    can have a neighborhood homeomorphic to an open set in \(\mathbb{R}^m\).

    Similarly, if \(n < m\), then there are not enough particles that can simultaneously move
    for an \(m\)-cube to exist in \(\DConf_n(\Gamma)\). So, \(n \ge m\). 
\end{proof}


Abrams proved the following Lemma in \cite{abrams2000configurationspaces}. We include 
this result with essentially the same proof for completeness.

\begin{lem}
    \label{lem:is_connected_0}
    If \(\DConf_n(\Gamma)\) is connected, then \(\Gamma\) has at least \(n + 1\) vertices
    and \(\Gamma\) is connected.
\end{lem}
\begin{proof}
    Suppose \(\DConf_n(\Gamma)\) is connected.
    If \(\Gamma\) has less than \(n + 1\) vertices, then \(\DConf_n(\Gamma)\) is either empty or
    consists of \(n!\) isolated points; so, \(\Gamma\) must have at least \(n+1\) vertices.

    Let \(v_1\) and \(v_2\) be distinct vertices in \(\Gamma\).
    To construct a path in \(\Gamma\) between \(v_1\) and \(v_2\), first let \(w_1, \ldots, w_{n-1}\)
    be \(n-1\) vertices in \(\Gamma\) distinct from \(v_1\) and \(v_2\).
    Next, since \(\DConf_n(\Gamma)\) is connected, there must be a path \(\gamma\)
    in \(\DConf_n(\Gamma)\) between the configurations
    \((v_1, w_1, \ldots, w_{n-1})\) and \((v_2, w_1, \ldots, w_{n-1})\).
    The path \(\gamma\) corresponds to a sequence of (potentially simultaneous) movements of particles on \(\Gamma\).
    In particular the particle at \(v_1\) must be able to eventually move to \(v_2\).
    The sequence of edges traveled by this particle gives a path in \(\Gamma\) between \(v_1\) and \(v_2\).
\end{proof}


This next lemma is the main tool we use to restrict the class of graphs that can be manifolds
without boundary. All later arguments are essentially repeated applications of this lemma
to different subgraphs of \(\Gamma\).

\begin{lem}
    \label{lem:is_surface_1}
    Suppose \(\DConf_n(\Gamma)\) is a \(2\)-manifold without boundary and let
    \(v_1\) and \(w_1\) be adjacent vertices in \(\Gamma\). For any collection
    of \(n-1\) vertices \(\{v_2, \cdots, v_n\}\) in \(\Gamma - \{v_1, w_1\}\)
    there exists exactly two edges \(e_1 = v_i w_2\) and \(e_2 = v_j w_3\) such that
    \begin{enumerate}[label=(\roman*)]
        \item \(v_i\) and \(v_j\) belong to \(\{v_2, \cdots, v_n\}\)
        \item \(w_2\) and \(w_3\) fall outside of \(\{w_1, v_1, \cdots, v_n\}\)
        \item exactly one of the following hold in Figure \ref{fig:lem:is_surface_1_1}.
    \end{enumerate}

\begin{figure}[h!]
    \centering
        \begin{enumerate*}[label=(\arabic*)]
            \item \label{fig:lem:is_surface_1_1:1}
            \begin{minipage}{.3\textwidth}
                \centering
                \(v_i = v_j\) \textit{and} \(w_2 \neq w_3\) \\
                \vspace{1em}
                \begin{tikzpicture}
                \node (v1) at (0,2) {\(v_1\)};
                \node (w1) at (0,0) {\(w_1\)};
                \draw (v1) -- (w1);
                \node (vi) at (3, 2) {\(v_i\)};
                \node (w2) at (2, 0) {\(w_2\)};
                \node (w3) at (4, 0) {\(w_3\)};
                \draw (vi) -- (w2) node[midway, left] {\(e_1\)};
                \draw (vi) -- (w3) node[midway, right] {\(e_2\)};
                \end{tikzpicture} 
            \end{minipage}
            \hspace{3em}

            \item \label{fig:lem:is_surface_1_1:2}
            \begin{minipage}{.3\textwidth}
                \centering
                \(w_2 = w_3\) \textit{and} \(v_i \neq v_j\) \\
                \vspace{1em}
                \begin{tikzpicture}
                    \node (v1) at (0,2) {\(v_1\)};
                    \node (w1) at (0,0) {\(w_1\)};
                    \draw (v1) -- (w1);
                    \node (vi) at (2, 2) {\(v_i\)};
                    \node (vj) at (4, 2) {\(v_j\)};
                    \node (w2) at (3, 0) {\(w_2\)};

                    \draw (vi) -- (w2) node[midway, left] {\(e_1\)};
                    \draw (vj) -- (w2) node[midway, right] {\(e_2\)};
                \end{tikzpicture}
            \end{minipage}
        \end{enumerate*}
    \caption{Lemma \ref{lem:is_surface_1} possibilities.}
    \label{fig:lem:is_surface_1_1}
\end{figure}
\end{lem}
\begin{proof}
    Let \(v_2, \cdots, v_n\) be \(n-1\) vertices in \(\Gamma - \{v_1, w_1\}\).   
    Now, put a particle at each of the vertices \(v_1, \cdots, v_n\).
    Since no particle exists at \(w_1\) and \(v_1\) is adjacent to \(w_1\), 
    the particle at \(v_1\) can travel from \(v_1\) to \(w_1\).
    This particles movement corresponds to a \(1\)-cube in \(\DConf_n(\Gamma)\).
    Since the configuration space is a \(2\)-manifold without boundary, 
    this \(1\)-cube must border exactly two distinct \(2\)-cubes.
    For this \(1\)-cube to border some \(2\)-cube,
    there must be another particle at some vertex \(v_i \in \{v_2, \ldots, v_n\}\) that can move to some vertex
    \(w_2 \not \in \{w_1, v_1, \ldots, v_n\}\).
    Furthermore, since this \(1\)-cube needs to border two \(2\)-cubes, 
    the movement of the particle from \(v_i\) to \(w_2\) cannot be the only other movement in addition to
    the movement of the particle from \(v_1\) to \(w_1\). 
    Let \(v_j\) and \(w_3\) be the vertices corresponding to this other movement.
    
    Notice that if \(v_i \neq v_j\) and \(w_2 \neq w_3\), then as the particle at \(v_1\) travels to \(w_1\), the particles
    at \(v_i\) and \(v_j\) can simultaneously travel to \(w_2\) and \(w_3\) respectively. These three simultaneous movements correspond
    to a \(3\)-cube in \(\DConf_n(\Gamma)\) with the point \((v_1, \cdots, v_n)\) as a corner.
    Since \(\DConf_n(\Gamma)\) is assumed to be a \(2\)-manifold and \(e_1\) and \(e_2\) are distinct, 
    the result follows.
\end{proof}

The next two lemmas require Lemma \ref{lem:is_surface_2} which is postponed until the end of the section
due to its length.

\begin{lem}
\label{lem:is_surface_C}
Suppose \(\DConf_n(\Gamma)\) is a surface without boundary for some \(n \ge 3\)
and let \(v_1\) and \(w_1\) be adjacent vertices in \(\Gamma\).
If \(\Gamma - \{v_1, w_1\}\) contains a cycle \(C\), then \(C\) is an \(n\)-cycle and \(\Gamma - \{v_1, w_1\} = C\).
\end{lem}
\begin{proof}
Suppose \(\Gamma - \{v_1, w_1\}\) contains a cycle \(C\).
Suppose that \(C\) has \(r > n\) vertices and let \(w_2, v_2, \cdots, v_n, w_3\) be \(n+1\) distinct vertices in a path on \(C\).
Then, the edges \(v_2 w_2\) and \(v_n w_3\) contradict the result of 
Lemma \ref{lem:is_surface_1} applied to \(\{v_2, \cdots, v_n\}\) in \(\Gamma - \{v_1, w_1\}\).
Therefore, \(C\) must have at most \(n\) vertices.

Suppose instead that \(C\) has \(r < n\) vertices.
First note that since \(C\) must have at least \(3\) vertices to be a cycle, \(n\) must be strictly greater than \(3\).
By Lemma \ref{lem:is_surface_0}, there must exist at least \(n - r\) vertices in \(\Gamma - \{v_1, w_1\} - C\).
Let \(v_2, \cdots, v_n\) be \(n - 1\) vertices in \(\Gamma - \{v_1, w_1\}\) such that 
\(\{v_2, \cdots, v_{r + 1}\}\) are the vertices of \(C\) and \(\{v_{r + 2}, \cdots, v_n\}\) are not on \(C\).
Applying Lemma \ref{lem:is_surface_1} to \(\{v_2, \cdots, v_n\}\) in \(\Gamma - \{v_1, w_1\}\), 
we obtain two edges \(e_1 = v_i w_2\) and \(e_2 = v_j w_3\) where \(w_2, w_3 \not \in \{w_1, v_1, v_2, \cdots, v_n\}\).

First, suppose at least one of \(v_i\) or \(v_j\) is on \(C\). 
Without a loss of generality assume \(v_i\) is on \(C\). 
Now put particles on the vertices in \(\{w_2, v_1, \cdots, v_n\}\setminus\{v_i\}\).
As the particle at \(v_1\) travels to \(w_1\), 
there are three particles that can move to \(v_i\): 
the particle at \(w_2\) and the two particles at the vertices on \(C\) which are adjacent to \(v_i\).
These particles movements result in a book in the configuration space whose spine corresponds to the movement of the particle at \(v_1\) to \(w_1\).
Therefore neither \(v_i\) nor \(v_j\) can be on \(C\).

Suppose instead that neither \(v_i\) nor \(v_j\) are on \(C\).
Note that \(w_2\) and \(w_3\) cannot be on \(C\) either since 
\(\{v_2, \cdots, v_{m+1}\}\) are the vertices of \(C\),
and \(w_2\) and \(w_3\) do not belong to \(\{w_1, v_1, v_2, \cdots, v_n\}\).
Since \(n > 3\), Lemma \ref{lem:is_surface_2} guarantees that \(e_1\) and \(e_2\) belong to a cycle \(D\)
with at least one vertex \(u\) in \(\Gamma - \{w_1, w_2, v_1, \cdots, v_n\}\).

We claim that there cannot be any edge \(e = u_1 u_2\) such that \(u_1\) is on \(\Gamma - \{v_1, w_1\} - D\) and \(u_2\) is on \(D\).
To see this suppose there did exist such an edge \(e\).
Now put \(n\) particles on \(\Gamma\) so that 
one particle is on \(v_1\), one particle is on \(u_1\), two particles are on \(D - u_2\), 
and \(n - 3\) particles are on \(\Gamma - \{v_1, w_1\} - D - u_1\).
As the particle at \(v_1\) moves to \(w_1\), the particle at \(u_1\) can move to \(u_2\) or 
the one of the two particles on \(D\) at the vertices adjacent to \(u_2\) can move to \(u_2\).
These particles movements result in a book in the configuration space whose spine corresponds to the movement of the particle at \(v_1\) to \(w_1\).

% TODO we can reach a contradiction another way by
% showing that \Gamma - \{v_1, w_1\} = C + D.
% Then, put all n particles on C and D.
% Some movement must be possible so we must then have edges sticking out of the cycles onto v_1 w_1
% But no edge works.
Since \(\Gamma\) is connected but \(D\) is not connected to any vertex in \(\Gamma - \{v_1, w_1\} - D\),
there must exist at least one edge connecting an endpoint of \(v_1 w_1\) to \(D\).
Suppose \(e = u_1 u_2\) is such an edge where \(u_1\) is on \(v_1 w_1\) and \(u_2\) is on \(D\).
Put \(n\) particles on \(\Gamma\) so that \(2\) particles are on each endpoint of \(v_1 w_1\), 
\(2\) particles are on \(D - u_2\), 
\(r - 1\) particles are on \(C\), and \(n - r - 3\) particles are on \(\Gamma - \{v_1, w_1\} - D\).
Since \(C\) is an \(r\)-cycle and there are only \(r - 1\) particles on it, 
there exists one particle on \(C\) that can move to another vertex on \(C\).
As this particle moves, the particle at \(u_1\) or one of the two on \(D - u_2\) can move to \(u_2\).
These particles movements again result in a book whose spine corresponds to the movement of the particle on \(C\).
Since surfaces do not contain books, \(C\) must be an \(n\)-cycle.

To see that \(C\) must equal \(\Gamma - \{v_1, w_1\}\) suppose there exists some vertex \(u\) in \(\Gamma - \{v_1, w_1\} - C\).
Since \(C\) is an \(n\)-cycle, \(\Gamma - \{v_1, w_1\}\) has at least \(n + 1\) vertices.
Put particles on \(\Gamma\) so that \(1\) particle is at \(v_1\), \(1\) particle is at \(u\) and \(n-2\) particles are on a path in \(C\).
As seen in the argument for why \(C\) cannot have more than \(n\) vertices, these particles movements correspond to a \(3\)-cube in the configuration
space. Hence \(\Gamma - \{v_1, w_1\}\) must be an \(n\)-cycle.
\end{proof}

\begin{lem}
\label{lem:is_surface_Y}
Let \(v_1\) and \(w_1\) be adjacent vertices in \(\Gamma\) and suppose \(n \ge 3\).
If \(\DConf_n(\Gamma)\) is a \(2\)-manifold without boundary and \(\Gamma - \{v_1, w_1\}\) does not contain a cycle, then \(\Gamma - \{v_1, w_1\}\) is the \(Y\)-graph.
\end{lem}
\begin{proof}
Suppose that \(\Gamma - \{v_1, w_1\}\) does not contain a cycle.
Lemma \ref{lem:is_surface_0} guarantees that \(\Gamma - \{v_1, w_1\}\) contains
at least \(n\) vertices.
Let \(\{v_2, \cdots, v_n\}\) be \(n-1\) vertices in \(\Gamma - \{v_1, w_1\}\)
and apply Lemma \ref{lem:is_surface_1} to \(\{v_2, \cdots, v_n\}\) in \(\Gamma - \{v_1, w_1\}\).
We then obtain two edges we obtain two edges \(e_1 = v_i w_2\) and \(e_2 = v_j w_3\).
Since \(\Gamma - \{v_1, w_1\}\) does not contain a cycle, Lemma \ref{lem:is_surface_2} guarantees that
\(e_1\) and \(e_2\) belong to some \(Y\)-graph \(Y\) and \(n = 3\).

First we show that there can not exist any other vertices in \(\Gamma - \{v_1, w_1\}\) other than those in \(Y\).
Suppose there exists some vertex \(u\) in \(\Gamma - \{v_1, w_1\} - Y\). 
Let \(x\) be the center vertex in \(Y\) that is incident to \(3\) edges.
Now, put 1 particle at \(v_1\), \(1\) particle at \(x\) and \(1\) particle at \(u\).
As the particle at \(v_1\) moves to \(w_1\), the particle at \(x\) can move
to 3 distinct vertices while the particle at \(u\) remains fixed.
These particles' movements result in a book in the configuration space
whose spine corresponds the movement of the particle at \(v_1\) to \(w_1\).
So, every vertex in \(\Gamma - \{v_1, w_1\}\) must belong to \(Y\).

Next we show there cannot be any other edges between vertices in \(\Gamma - \{v_1, w_1\}\)
other than those in \(Y\). Again, let \(x\) be the center vertex in \(Y\).
Since \(x\) is connected to every vertex in \(\Gamma - \{v_1, w_1\}\), if 
there are any additional edges, they must be between vertices other than \(x\).
So, suppose there exists some edge between two vertices \(u_1\) and \(u_2\) in \(\Gamma - \{x, v_1, w_1\}\).
Immediately we reach a contradiction as the vertices in \(\{u_1, x, u_2\}\) now form a \(3\)-cycle.
Therefore, \(\Gamma - \{v_1, w_1\} = Y\).
\end{proof}

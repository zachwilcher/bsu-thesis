\section{Higher Dimensional Manifolds}



\begin{lem}
  \label{lem:maybe-true}
  Let \(m \ge 2\) and suppose \(\DConf_n(\Gamma)\) is an \(m\)-manifold without boundary.
  There exists a surface without boundary \(X\) such that \(\DConf_{n - (m-2)}(\Gamma - V(M)) \cong X\) for any \((m-2)\)-matching \(M\) in \(\Gamma\).
\end{lem}

\begin{proof}
  Suppose that \(\DConf_{n - (m-2)}(\Gamma - V(M_0))\) is not a surface for some \((m-2)\)-matching \(M_0\) in \(\Gamma\).
  Then, there exists a configuration \(x \in \DConf_{n - (m - 2)}(\Gamma - V(M_0))\) such that every neighborhood of \(x\) is not homeomorphic to \(\mathbb{R}^2\).
  Place \(m - 2\) particles on the interior of each edge of \(M_0\) in \(\Gamma\) and let \(y\) denote this configuration.
  Define \(p = (x,y)\).
  Since \(p \in \DConf_n(\Gamma)\) and \(\DConf_n(\Gamma)\) is an \(m\)-manifold,
  there exists a neighborhood \(U\) of \(p\) such that \(U \cong \mathbb{R}^m\).

  


  % TODO: show the surface is unique.
\end{proof}

\begin{thm}
  Let \(m \ge 2\). If \(\DConf_n(\Gamma)\) is a connected \(m\)-manifold without boundary, 
  then \(\DConf_n(\Gamma)\) is one of the following.
\begin{enumerate}
  \item \(\DConf_m(K_{2m+1}) \cong \DConf_{m+1}(K_{2m+1})\)
  \item \(\DConf_m(K_{m+1, m+1}) \cong \DConf_{m+2}(K_{m+1, m+1})\)
  \item \(\DConf_{m+1}(K_{m, m+2})\)
\end{enumerate}
\end{thm}

\begin{proof}
  Let \(M\) be an \((m-2)\) matching in \(\Gamma\).
  By Lemma \ref{lem:maybe-true}, \(\DConf_{n - (m-2)}(\Gamma - V(M))\) is a surface without boundary.
  So, as shown in the previous section there are \(5\) possible \((n - (m-2), \Gamma - V(M))\) pairs.
  \begin{enumerate}
    \item If \((n - (m - 2), \Gamma - V(M)) = (2, K_5)\), 
    then \(n\) must be \(m\) and 
    by Lemma \ref{lem:complete-graph-is-special}, \(\Gamma\) must be \(K_{2m+1}\).

    \item If \((n - (m - 2), \Gamma - V(M)) = (3, K_5)\), 
    then \(n\) must be \(m + 1\) and
    by Lemma \ref{lem:complete-graph-is-special}, \(\Gamma\) must be \(K_{2m+1}\).

    \item If \((n - (m - 2), \Gamma - V(M)) = (2, K_{3,3})\), 
    then \(n\) must be \(m\) and
    by Lemma \ref{lem:bipartite-graph-is-special}, \(\Gamma\) must be \(K_{m+1, m+1}\).

    \item If \((n - (m - 2), \Gamma - V(M)) = (4, K_{3,3})\), 
    then \(n\) must be \(m + 2\) and
    by Lemma \ref{lem:bipartite-graph-is-special}, \(\Gamma\) must be \(K_{m+1, m+1}\).

    \item If \((n - (m - 2), \Gamma - V(M)) = (3, K_{2,4})\), 
    then \(n\) must be \(m + 1\) and
    by Lemma \ref{lem:bipartite-graph-is-special}, \(\Gamma\) must be \(K_{m, m+2}\).

  \end{enumerate}
\end{proof}

In \cite{abrams2000configurationspaces}, Abrams showed the following.
\begin{prop}
If \(m > 2\) then \(\DConf_m(K_{(m+1), (m+1)})\) and \(\DConf_{m}(K_{(2m+1)})\) are not \(m\)-manifolds.
\end{prop}
Applying his duality theorem, it follows that
\(\DConf_{m+1}(K_{m+1, m+1})\) and \(\DConf_{m+1}(K_{2m+1})\)
are not \(m\) manifolds either.
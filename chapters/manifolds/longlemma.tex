\subsection{A painful and long lemma}
\begin{lem}
    \label{lem:manifolds_2}
    Let \(\{v_\alpha\}_{\alpha=m}^{n}\) be a collection of \(n-m+1\) vertices in 
    \(\Gamma - \{v_\alpha w_\alpha\}_{\alpha=1}^{m-1}\).
The edges \(e_1\) and \(e_2\) guaranteed from applying Lemma \ref{lem:manifolds_1} 
to \(\{v_\alpha\}_{i=m}^{n}\) in the graph \(\Gamma - \{v_\alpha w_\alpha\}_{i=\alpha}^{m-1}\) satisfy exactly one of the following:
\begin{enumerate}[label=(\roman*)]
        \item \(n = m+1\) and these edges belong to a \(Y\)-graph in \(\Gamma - \{v_i, w_i\}_{i=1}^{m-1}\) with two vertices not in \(\{v_i\}_{i=1}^n\).
        \item \(e_1\) and \(e_2\) belong to a \(3\)-cycle in \(\Gamma - \{v_i, w_i\}_{i=1}^{m-1}\) with exactly two vertices not in \(\{v_i\}_{i=1}^n\).
        \item \(e_1\) and \(e_2\) belong to an \(r\)-cycle in \(\Gamma - \{v_i, w_i\}_{i=1}^{m-1}\) with exactly one vertex not in \(\{v_i\}_{i=1}^n\) and \(3 \le r \le (n-m)+2\).
\end{enumerate}

\end{lem}


\begin{proof}
    The argument proceeds by cases on the possibilities for the edges \(e_1\) and \(e_2\).
    
    \textbf{Case 1:} \(v_i = v_j\) and \(u_1 \neq u_2\) (See case \ref{fig:lem:manifolds_1_1} in Figure \ref{fig:lem:manifolds_1}). 
    In this case, \(e_1 = v_i u_1\) and \(e_2 = v_i u_2\). 
    Let \(v_k\) be some vertex in \(\{v_\alpha\}_{i=m}^n\setminus\{v_i\}\).
    Applying Lemma \ref{lem:manifolds_1} to \(\{u_1\}\cup\{v_\alpha\}_{\alpha=m}^n\setminus\{v_k\}\) in \(\Gamma - \{v_\alpha, w_\alpha\}_{\alpha=1}^{m-1}\),
    we obtain two new edges \(e_3\) and \(e_4\).
    Since \(e_2\) already connects \(v_i\) to \(u_2\) and there must be exactly two edges
    with exactly one endpoint in \(\{u_1\}\cup\{v_\alpha\}_{\alpha=m}^n\setminus\{v_k\}\),
    one of \(e_3\) or \(e_4\) is \(e_2\). 
    Without a loss of generality suppose \(e_4 = e_2\).
    Since \(e_3\) must share an endpoint with \(e_4 = e_2\), it follows that \(e_3\) has either \(v_i\) as an endpoint or \(u_2\) as an endpoint (See Figure \ref{fig:lem:is_surface_2_0}).
    \begin{figure}[h!]
        \centering
        \begin{tikzpicture}
            \node (v1) at (0,2) {\(v_1\)};
            \node (w1) at (0,0) {\(w_1\)};
            \draw (v1) -- (w1);
            \node (vi) at (3, 2) {\(v_i\)};
            \node (w2) at (2, 0) {\(w_2\)};
            \node (w3) at (4, 0) {\(w_3\)};
            \draw (vi) -- (w2) node[midway, left] {\(e_1\)};
            \draw (vi) -- (w3) node[midway, right] {\(e_2\)};
            
            \node (vk) at (5, 2) {\(?\)};
            \draw (vi) -- (vk) node[midway, above] {\(e_3\)};
        \end{tikzpicture}
        \quad\quad
        \begin{tikzpicture}
            \node (v1) at (0,2) {\(v_1\)};
            \node (w1) at (0,0) {\(w_1\)};
            \draw (v1) -- (w1);
            \node (vi) at (3, 2) {\(v_i\)};
            \node (w2) at (2, 0) {\(w_2\)};
            \node (w3) at (4, 0) {\(w_3\)};
            \node (q) at (6, 0) {\(?\)};
            \draw (vi) -- (w2) node[midway, left] {\(e_1\)};
            \draw (vi) -- (w3) node[midway, right] {\(e_2\)};
            \draw (q) -- (w3) node[midway, below] {\(e_3\)};
            
        \end{tikzpicture}
        \caption{Possibilities if \(v_i = v_j\)}
        \label{fig:lem:is_surface_2_0}
    \end{figure}

    Lemma \ref{lem:manifolds_1} guarantees relevant two facts about the edges \(e_1\), \(e_2\), and \(e_3 = e_4\).
    % TODO continue fixing from here.
    \begin{enumerate}
        \item \(e_1\) and \(e_2\) are the only edges with exactly one endpoint in \(\{v_\alpha\}_{\alpha=1}^{m-1}\).
        \item \(e_3\) and \(e_4 = e_2\) are the only edges with exactly one endpoint in \(\{w_2, v_2, \cdots, v_n\}\setminus\{v_k\}\).
    \end{enumerate}
    We claim that if \(e_3\) has \(v_i\) as an endpoint, then \(v_k\) must be the other endpoint of \(e_3\)
    (see the graph on the left in Figure \ref{fig:lem:is_surface_2_1}).
    To verify this claim, suppose \(e_3\) has \(v_i\) as an endpoint and let \(v^*\) be the other endpoint of \(e_3\).
    Since \(e_3 \neq e_1\) and \(e_3 \neq e_2\) but \(e_3\) has \(v_i\) as an endpoint, the first fact above 
    guarantees that \(v^* \in \{v_2, \cdots, v_n\}\).
    The second fact guarantees that \(v^* \not \in \{w_2, v_2, \cdots, v_n\}\setminus\{v_k\}\).
    Hence,
    \[
        v^* \in \left(V(\Gamma - \{v_1, w_1\})\setminus(\{w_2, v_2, \cdots, v_n\}\setminus\{v_k\})\right) \cap \{v_2, \cdots, v_n\} = \{v_k\}.
    \]

    Moreover, if \(e_3\) has \(v_i\) as an endpoint, 
    then since \(v_k\) was arbitrary, 
    every vertex in \(\{v_2, \cdots, v_n\}\setminus\{v_i\}\) is adjacent to \(v_i\).

    So, if \(n > 3\) then there are two vertices say \(v_k\) and \(v_r\) in the set \(\{v_2, \cdots, v_n\}\) that are distinct but adjacent to \(v_i\).
    Put particles at each vertex in the set \(\{w_2, v_1, \cdots, v_n\}\setminus\{v_i\}\).
    As the particle at \(v_1\) moves to \(w_1\), any one particle in the set \(\{w_2, v_k, v_r\}\) can move to \(v_i\).
    These particles movements result in a book in the configuration space whose spine corresponds to the movement of the particle at \(v_1\) to \(w_1\).
    Therefore if \(e_3\) has \(v_i\) as an endpoint, then \(e_1\) and \(e_2\) belong to a \(Y\)-graph \textit{and} \(n = 3\).

    \begin{figure}[h!]
        \centering
        \begin{tikzpicture}
            \node (v1) at (0,2) {\(v_1\)};
            \node (w1) at (0,0) {\(w_1\)};
            \draw (v1) -- (w1);
            \node (vi) at (3, 2) {\(v_i\)};
            \node (w2) at (2, 0) {\(w_2\)};
            \node (w3) at (4, 0) {\(w_3\)};
            \draw (vi) -- (w2) node[midway, left] {\(e_1\)};
            \draw (vi) -- (w3) node[midway, right] {\(e_2\)};
            
            \node (vk) at (5, 2) {\(v_k\)};
            \draw (vi) -- (vk) node[midway, above] {\(e_3\)};
        \end{tikzpicture}
        \quad\quad
        \begin{tikzpicture}
            \node (v1) at (0,2) {\(v_1\)};
            \node (w1) at (0,0) {\(w_1\)};
            \draw (v1) -- (w1);
            \node (vi) at (3, 2) {\(v_i\)};
            \node (w2) at (2, 0) {\(w_2\)};
            \node (w3) at (4, 0) {\(w_3\)};
            \draw (vi) -- (w2) node[midway, left] {\(e_1\)};
            \draw (vi) -- (w3) node[midway, right] {\(e_2\)};
            \draw (w2) -- (w3) node[midway, below] {\(e_3\)};
            
        \end{tikzpicture}
        \caption{Possibilities if \(v_i = v_j\)}
        \label{fig:lem:is_surface_2_1}
    \end{figure}

    Suppose now that \(e_3\) has \(w_3\) as an endpoint.
    We claim that \(e_3\) must have \(w_2\) as its other endpoint
    (see the graph on the right in Figure \ref{fig:lem:is_surface_2_1}).
    To verify this claim, let \(w_*\) be the other endpoint of \(e_3\).
    Since \(w_3 \not \in \{w_2, v_2, \cdots, v_n\}\setminus\{v_k\}\),
    the second fact above guarantees that \(w_*\) must belong to \(\{w_2, v_2, \cdots, v_n\}\setminus\{v_k\}\).
    Since \(e_3 \neq e_1\) and \(e_3 \neq e_2\), the first fact above guarantees that \(w_*\)
    cannot belong to \(\{v_2, \cdots, v_n\}\).
    The only possibility left is that \(w_* = w_2\).

    Therefore, if \(e_3\) has \(w_3\) as an endpoint, then \(e_1\) and \(e_2\) belong to a \(3\)-cycle with two vertices
    not in \(\{v_2, \cdots, v_n\}\).

    \textbf{Case 2:} \(v_i \neq v_j\) (See \ref{fig:lem:is_surface_1_1:2} in Figure \ref{fig:lem:is_surface_1_1}).
    In this case, \(e_1 = v_i w_2\) and \(e_2 = v_j w_2\).
    Applying Lemma \ref{lem:is_surface_1} to \(\{w_2, v_2, \cdots, v_n\}\setminus\{v_j\}\) in \(\Gamma - \{v_1, w_1\}\),
    we again obtain two edges \(e_3\) and \(e_4\).
    Since \(e_2\) already connects \(v_j\) to \(w_2\)
    and there must be exactly two edges with exactly one endpoint in \(\{w_2, v_2, \cdots, v_n\}\setminus\{v_j\}\), 
    one of \(e_3\) or \(e_4\) is \(e_2\).
    Without a loss of generality suppose \(e_4 = e_2\).
    Since \(e_3\) must share an endpoint with \(e_4 = e_2\), it follows that \(e_3\)
    has either \(w_2\) or \(v_j\) as an endpoint
    (see Figure \ref{fig:lem:is_surface_2_2}).
    \begin{figure}[h!]
        \centering
        \begin{tikzpicture}
            \node (v1) at (0,2) {\(v_1\)};
            \node (w1) at (0,0) {\(w_1\)};
            \draw (v1) -- (w1);
            \node (vi) at (2, 2) {\(v_i\)};
            \node (vj) at (4, 2) {\(v_j\)};
            \node (w2) at (3, 0) {\(w_2\)};
            \draw (vi) -- (w2) node[midway, left] {\(e_1\)};
            \draw (vj) -- (w2) node[midway, right] {\(e_2\)};

            \node (q) at (5, 0) {\(?\)};
            \draw (w2) -- (q) node[midway, below] {\(e_3\)};
        \end{tikzpicture}
        \quad\quad
        \begin{tikzpicture}
            \node (v1) at (0,2) {\(v_1\)};
            \node (w1) at (0,0) {\(w_1\)};
            \draw (v1) -- (w1);
            \node (vi) at (2, 2) {\(v_i\)};
            \node (vj) at (4, 2) {\(v_j\)};
            \node (w2) at (3, 0) {\(w_2\)};
            \draw (vi) -- (w2) node[midway, left] {\(e_1\)};
            \draw (vj) -- (w2) node[midway, right] {\(e_2\)};

            \node (q) at (6, 2) {\(?\)};
            \draw (vj) -- (q) node[midway, above] {\(e_3\)};
        \end{tikzpicture}
        \caption{Possibilities if \(v_i \neq v_j\)}
        \label{fig:lem:is_surface_2_2}
    \end{figure}

    Since \(e_1\) and \(e_2\) are the only edges connecting a vertex in \(\{v_2, \cdots, v_n\}\) to \(w_2\) and \(e_3 \neq e_1\) and \(e_3 \neq e_2\),
    if \(e_3\) has \(w_2\) as endpoint, then \(e_3\) must connect \(w_2\) to some vertex \(w_3\) in \(\Gamma - \{v_1, w_1\}\) outside the set \(\{v_2, \cdots, v_n\}\).

    Suppose that \(e_3\) has \(w_2\) but \(n > 3\),
    then there exists some \(v_k\) in \(\{v_2, \cdots, v_n\}\) distinct from \(v_i\) and \(v_j\). 
    Let \(w_3\) be the other endpoint of \(e_3\) as before and put particles at the vertices \(\{w_3, v_1, \cdots, v_n\}\setminus\{v_k\}\). 
    As the particle at \(v_1\) moves to \(w_1\),
    any one particle in the set \(\{v_i, v_j, w_3\}\) can move to \(w_2\). 
    These particles movements results in a book in the configuration space whose spine corresponds to the movement of the particle at \(v_1\) to \(w_1\). 
    Hence, if \(e_3\) has \(w_2\) as an endpoint, 
    then \(n=3\) \textit{and} \(e_1\) and \(e_2\) both belong to a \(Y\)-graph with two vertices not in \(\{v_2, \cdots, v_n\}\)
    (see the graph on the left in Figure \ref{fig:lem:is_surface_2_3}).

    \begin{figure}[h!]
        \centering
        \begin{tikzpicture}
            \node (v1) at (0,2) {\(v_1\)};
            \node (w1) at (0,0) {\(w_1\)};
            \draw (v1) -- (w1);
            \node (vi) at (2, 2) {\(v_i\)};
            \node (vj) at (4, 2) {\(v_j\)};
            \node (w2) at (3, 0) {\(w_2\)};
            \draw (vi) -- (w2) node[midway, left] {\(e_1\)};
            \draw (vj) -- (w2) node[midway, right] {\(e_2\)};

            \node (u) at (5, 0) {\(w_3\)};
            \draw (w2) -- (u) node[midway, below] {\(e_3\)};
        \end{tikzpicture}
        \quad\quad
        \begin{tikzpicture}
            \node (v1) at (0,2) {\(v_1\)};
            \node (w1) at (0,0) {\(w_1\)};
            \draw (v1) -- (w1);
            \node (vi) at (2, 2) {\(v_i\)};
            \node (vj) at (4, 2) {\(v_j\)};
            \node (w2) at (3, 0) {\(w_2\)};
            \draw (vi) -- (w2) node[midway, left] {\(e_1\)};
            \draw (vj) -- (w2) node[midway, right] {\(e_2\)};

            \node (vk) at (6, 2) {\(v_k\)};
            \draw (vk) -- (vj) node[midway, above] {\(e_3\)};
        \end{tikzpicture}
        \caption{Possibilities if \(v_i \neq v_j\)}
        \label{fig:lem:is_surface_2_3}
    \end{figure}

    Finally, suppose that \(e_3\) has \(v_j\) as an endpoint. 
    Since \(e_1\) and \(e_2\) are the only edges with exactly one endpoint in \(\{v_2, \cdots, v_n\}\),
    the other endpoint of \(e_3\) must belong to \(\{v_2, \cdots, v_n\}\setminus\{v_j\}\).
    Let \(v_k\) be this endpoint and notice that \(\{v_i, w_2, v_j, v_k\}\) cannot belong to a \(Y\)-graph
    (see the graph on the right in Figure \ref{fig:lem:is_surface_2_3}).
    
    We claim that the vertices in \(\{v_i, w_2, v_j, v_k\}\) belong to a cycle.
    If \(n = 3\), the only possible choice for \(v_k\) is \(v_i\), 
    meaning the vertices in \(\{w_2, v_2, v_3\}\) form a \(3\)-cycle in \(\Gamma - \{v_1, w_1\}\).
    So suppose \(n > 3\) and relabel the vertices in \(\{v_i, w_2, v_j, v_k\}\) such that 
    \(v_i = u_1, w_2 = u_2, v_j = u_3, v_k = u_4\) (see Figure \ref{fig:lem:is_surface_2_4}).
    \begin{figure}[h!]
        \centering
        \begin{tikzpicture}
            \node (v1) at (0,2) {\(v_1\)};
            \node (w1) at (0,0) {\(w_1\)};
            \draw (v1) -- (w1);

            \node (u1) at (2, 1) {\(u_1\)};
            \node (u2) at (3, 0) {\(u_2\)};
            \node (u3) at (4, 1) {\(u_3\)};
            \node (u4) at (3, 2) {\(u_4\)};
            \draw (u1) -- (u2) node[midway, below left] {\(e_1\)};
            \draw (u2) -- (u3) node[midway, below right] {\(e_2\)};
            \draw (u3) -- (u4) node[midway, above right] {\(e_3\)};
        \end{tikzpicture}
        \caption{Partial cycle if \(v_i \neq v_j\) and \(e_3\) has \(v_j\) as an endpoint.}
        \label{fig:lem:is_surface_2_4}
    \end{figure}

    We now inductively construct a path in \(\Gamma\)
    consisting of vertices in \(\{w_2, v_2, \cdots, v_n\}\) starting at \(u_1\).

    Suppose \(u_{i-1}, u_i, u_{i+1}\) are 3 distinct vertices in \(\{w_2, v_2, \cdots, v_n\}\) such that
    \(u_{i-1}\) is connected to \(u_{i-2}\) by the edge \(e_{i-2}\)
    and \(u_i\) is connected to \(u_{i-1}\) by the edge \(e_{i-1}\) for some \(i > 3\).
    Apply Lemma \ref{lem:is_surface_1} to \(\{w_2, v_2, \cdots, v_n\}\setminus\{u_i\}\) in \(\Gamma - \{v_1, w_1\}\) and obtain
    two edges \(e\) and \(e'\).
    Since \(e\) and \(e'\) are 
    the only edges with exactly one endpoint in \(\{w_2, v_2, \cdots, v_n\}\setminus\{u_i\}\)
    and  \(u_i\) is already adjacent to \(u_{i-1}\), one of \(e\) or \(e'\) must
    be the edge \(e_{i-1}\).
    Without a loss of generality suppose \(e' = e_{i-1}\) and define \(e_i = e\).

    Since \(e_{i-1}\) and \(e_{i}\) must share an endpoint,
    one of the endpoints of \(e_i\) is \(u_{i-1}\) or \(u_i\). 
    Let \(u_*\) be the other endpoint of \(e_i\) (see Figure \ref{fig:lem:is_surface_2_5}).

    \begin{figure}[h!]
        \centering
        \begin{tikzpicture}
            \node (v1) at (0,2) {\(v_1\)};
            \node (w1) at (0,0) {\(w_1\)};
            \draw (v1) -- (w1);
            \node (u1) at (2, 2) {\(u_{i-2}\)};
            \node (u2) at (3, 0) {\(u_{i-1}\)};
            \node (u3) at (4, 2) {\(u_{i}\)};
            \draw (u1) -- (u2) node[midway, left] {\(e_{i-2}\)};
            \draw (u2) -- (u3) node[midway, right] {\(e_{i-1}\)};

            \node (q) at (5, 0) {\(u_*\)};
            \draw (u2) -- (q) node[midway, below] {\(e_i\)};
        \end{tikzpicture}
        \quad\quad
        \begin{tikzpicture}
            \node (v1) at (0,2) {\(v_1\)};
            \node (w1) at (0,0) {\(w_1\)};
            \draw (v1) -- (w1);
            \node (u1) at (2, 2) {\(u_{i-2}\)};
            \node (u2) at (3, 0) {\(u_{i-1}\)};
            \node (u3) at (4, 2) {\(u_{i}\)};
            \draw (u1) -- (u2) node[midway, left] {\(e_{i-2}\)}; 
            \draw (u2) -- (u3) node[midway, right] {\(e_{i-1}\)};

            \node (q) at (6, 2) {\(u_*\)};
            \draw (u3) -- (q) node[midway, above] {\(e_i\)};
        \end{tikzpicture}
        \caption{Possibilities for \(e_i\)}
        \label{fig:lem:is_surface_2_5}
    \end{figure}

    Suppose \(e_i\) has \(u_{i-1}\) as an endpoint.
    Since \(e_i\) cannot share both its endpoints with \(e_{i-1}\), and
    \(e_i\) has exactly one endpoint in \(\{w_2, v_2, \cdots, v_n\}\setminus\{u_i\}\),
    it follows that \(u_*\) must fall outside of \(\{w_2, v_2, \cdots, v_n\}\).
    Put particles at each vertex in the set \(\{u_*, w_2, v_1, v_2, \cdots, v_n\}\setminus\{u_{i-2}, u_i\}\).
    Since \(i > 3\), as the particle at \(v_1\) travels to \(w_1\),
    the particle at \(u_{i-3}\) can travel to \(u_{i-2}\) simultaneously as the particle
    at \(u_{i-1}\) can travel to \(u_i\).
    These particles movements result in a \(3\)-cube in the configuration space.
    Therefore \(e_i\) cannot have \(u_{i-1}\) as an endpoint.
    
    Suppose instead that \(e_i\) has \(u_i\) as an endpoint.
    Since \(e_i\) must have exactly one endpoint in \(\{w_2, v_2, \cdots, v_n\}\setminus\{u_i\}\) and \(u_* \neq u_i\),
    it follows that \(u_*\) must belong to \(\{w_2, v_2, \cdots, v_n\}\setminus\{u_i, u_{i-1}\}\).
    Define \(u_{i+1} = u_*\).
    This completes the inductive step.
    
    Notice that each \(u_i\) belongs to the finite set \(\{w_2, v_2, \cdots, v_n\}\)
    and \(u_i\) is connected to \(u_{i-1}\) by the edge \(e_{i-1}\) for all \(i > 1\).
    We claim this path must eventually return to \(u_1\) forming a cycle.

    Let \(r\) be the largest integer such that \(u_r \not \in \{u_1, \cdots, u_{r-1}\}\).
    The edge \(e_r\) must connect \(u_r\) to some vertex \(u_k \in \{u_1, \cdots, u_{r-1}\}\)
    otherwise we contradict that \(r\) is maximal.
    Suppose \(k > 1\), then \(e_{k-1}\), \(e_k\), and \(e_r\) are three distinct
    edges sharing \(u_k\) as a common endpoint (see Figure \ref{fig:lem:is_surface_2_6}).

    \begin{figure}[h!]
        \centering
        \begin{tikzpicture}
            \node (ukm1) at (0,0) {\(u_{k-1}\)};
            \node (uk) at (2,0) {\(u_k\)};
            \node (ukp1) at (4,0) {\(u_{k+1}\)};
            \node (um) at (2,2) {\(u_r\)};

            \draw (ukm1) -- (uk) node[midway, below] {\(e_{k-1}\)};
            \draw (uk) -- (ukp1) node[midway, below] {\(e_k\)};
            \draw (uk) -- (um) node[midway, left] {\(e_r\)};
        \end{tikzpicture}
        \caption{Three edges sharing \(u_k\) as a common endpoint.}
        \label{fig:lem:is_surface_2_6}
    \end{figure}

    Place particles at each vertex in the set \(\{w_2, v_1, v_2, \cdots, v_n\}\setminus\{u_k\}\).
    As the particle at \(v_1\) travels to \(w_1\),
    each particle at \(u_{k-1}\), \(u_{k+1}\), and \(u_r\) can travel to \(u_k\).
    This results in a book in the configuration space whose spine corresponds to the movement of the particle at \(v_1\) to \(w_1\).
    Therefore \(k = 1\), meaning \(\{u_1, u_2, \cdots, u_r\}\) forms an \(r\)-cycle for some \(4 \le r \le n\)
    containing one vertex \(w_2\) not in \(\{v_2, \cdots, v_n\}\).
\end{proof}

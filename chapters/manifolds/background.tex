\section{Background}
In his thesis \cite{abrams2000configurationspaces}, Abrams asked:
For which \(n\) and \(\Gamma\) is \(\DConf_n(\Gamma)\) homeomorphic to a closed manifold?
He establishes a partial answer to this question in Corollary 5.8: 
if \(\Gamma\) is a connected graph
without loops and \(\DConf_n(\Gamma)\) is homeomorphic to a closed
\(n\)-dimensional manifold, then either \(n = 1\) and \(\Gamma\) is a cycle, or
\(n = 2\) and \(\Gamma\) is \(K_5\) or \(K_{3,3}\).

Advised by Peter Haskell, Praphat Fernandes in his undergraduate thesis
demonstrates that if \(\Gamma\) is a simple graph and \(\DConf_2(\Gamma)\)
is a \(2\)-psuedomanifold without boundary the following are true.
\begin{enumerate}
    \item If \(\Gamma\) contains a valence 4 vertex, then \(\Gamma\) is \(K_5\).
    \item If every vertex in \(\Gamma\) has valence 3, then \(\Gamma\) is \(K_{3,3}\).
\end{enumerate}
This recovers Abrams' result about when \(\DConf_2(\Gamma)\) 
is a \(2\)-manifold, as Praphat also shows that
if \(\Gamma\) is simple and \(\DConf_2(\Gamma)\) is a 2-psuedomanifold without boundary,
then the valence of every vertex in \(\Gamma\) is between 3 and 4,
and \(\DConf_2(\Gamma)\) is a 2-manifold. Also advised by Peter Haskell, Molly Ison in \cite{ison2005two} 
extends Fernandes' work in her masters thesis to 2-psuedomanifolds with boundary.

In Example 4.3 in \cite{ko2012characteristics}, it is shown that \(\DConf_3(\Theta_4)\)
is a closed orientable surface of genus 3 by computing its fundamental group and using the
fact that discretized configuration spaces are \(K(\pi, 1)\) spaces.
This result is recovered in \cite{appiah2024algebraicstructurehyperbolicgraph}
with a strictly combinatorial argument.

In this chapter we provide a partial answer to Abrams question
by characterizing exactly which graph configuration spaces
are connected \(m\)-manifolds without boundary.
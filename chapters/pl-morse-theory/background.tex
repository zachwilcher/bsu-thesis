\section{Background}

This is a modified proof of a statement in \cite{appiah2024algebraicstructurehyperbolicgraph}.
\begin{thm}
\label{thm:descendinglinks}
Let \(X\) be a finite cubical complex,
\(f: X \rightarrow \mathbb{R}\) be a PL Morse function,
and \([a,b]\) be contained in the image of \(f\). 
For any \(c\) in the open interval \((a,b)\), if
\begin{enumerate}
    \item \(f^{-1}([a,c])\) is connected.
    \item \(f^{-1}((c,b))\) contains no vertices.
    \item \(f^{-1}(\{b\})\) is a set of vertices.
    \item The descending links of each vertex \(v\) in \(f^{-1}(\{b\})\) 
        is homotopically equivalent to \(n_v\) distinct points.
\end{enumerate}
then, \(f^{-1}[[a,b]]\) is homotopy equivalent to a wedge of \(f^{-1}[[a,c]]\)
with exactly \(\displaystyle r = \sum_{v \in f^{-1}(\{b\})} (n_v - 1)\) circles and consequently
\(\pi_1(f^{-1}[[a,b]]) = \pi_1(f^{-1}[a,c])\,*\,\mathbb{F}_r\)
where \(\mathbb{F}_r\) is a free group on \(r\) generators.
\end{thm}

\begin{proof}
We first consider the case when there is only one vertex \(v\) in \(f^{-1}(\{b\})\).
Let the descending link of \(v\) be the disjoint spaces \(\{A_1, ..., A_{n_v}\}\). 

By Proposition 2.7 in \cite{Bestvina2008PL}, \(f^{-1}([c,b])\) is homotopically equivalent
rel \(f^{-1}(\{a\})\) to the cone on the descending links of \(v\) attached to \(v\). 

Since \(f^{-1}([a,c])\) is path connected, 
we can find paths from a point in each \(A_i\) to some \(p\) in \(f^{-1}(\{c\}) \cap A_1\).
Let \(Q\) be the union of the cone and these paths.

% Visually this next line is clear but I should say more.
Notice \(Q\) homotopically equivalent to a wedge of \(n_v - 1\) circles.

Since each added path in \(Q\) contracts to \(p\), we can retract \(Q \cap f^{-1}([a,c])\) to \(p\).

% I should be more explicit about how the fundamental group of the intersection is null homotopic
% and how \pi_1 of a wedge of circles is just a free group on how many circles there are.
The Seifert-Van Kampen theorem guarantees that \(\pi_1(f^{-1}([a,b]))\) is just \(\pi_1(f^{-1})([a,c]) * \mathbb{F}_{n_v - 1}\).

When there are multiple points in \(f^{-1}(\{b\})\), as mentioned in \cite{Bestvina2008PL}
the cones on each point \(v \in f^{-1}(\{b\})\) get attached to each \(v\) in a pairwise disjoint fashion.
Similar to before, by finding paths through \(f^{-1}([a,c])\) we can connect together each cone
resulting in a wedge of \(r = \sum_{v} (n_v - 1)\) circles.
The Seifert-Van Kampen theorem then guarantees that
\(\pi_1(f^{-1}([a,b])) = \pi_1(f^{-1}([a,c])) * \mathbb{F}_r\).

\end{proof}